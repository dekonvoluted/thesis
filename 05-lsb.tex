\chapter{LSB Flame Characteristics}

% The primary goal is as follows:

% Investigate the behavior of the LSB at GT conditions.
%  a. Flame characteristics and behavior with velocity at high pressure conditions.
%  b. Behavior with respect to other flow and geometric parameters.
%  c. Flame structure at high pressure conditions. Requires CH PLIF.

% Key questions:
% a. At atmospheric pressure, the flame standoff distance does not vary at all with increasing velocity.
%    Is this still true when approaching closer to gas turbine operating conditions?
%    What does that say about the model used to explain this behavior?

% b. If velocity has no effect on the flame characteristics, what about other flow parameters?
%    What effect does the preheat temperature have on the flame?
%    What happens to the flame at very high pressures?
%    How does the flame look at rich and lean operating conditions?
%    Does changing the swirler have an effect on the flame?

% c. How does the flame structure look at low and high pressure conditions?
%    Does this provide any insight into the flame behavior at these conditions?
%    What physical processes may be responsible for that?

In Chapter FIXME 2, we introduced the salient features of the Low Swirl Burner (LSB) flow field and discussed the mechanisms by which the LSB flame is stabilized.
Further, various characteristics of the LSB flame that can be measured from flame images were outlined.
To recapitulate, these are the flame location, flame shape and the flame structure.
The first two are quantified by the flame standoff distance, \(X_f\), and the flame angle, \(\theta_f\), respectively.

In the same chapter, we introduced the four flow parameters that describe an operating condition for the LSB --- the combustor pressure, \(p\), the preheat temperature, \(T\), the mass-averaged inlet velocity (also called the reference velocity, \(U_0\), and the equivalence ratio of the premixed reactants, \(\phi\).
We further introduced a geometric parameter --- the angle of the vanes of the swirler, \(\alpha\), which affects the amount of swirl present in the flow field.

The LSB flame is imaged over a range of operating conditions and the effect of flow and geometric parameters on the reacting flow field is investigated.
This results of the investigation are presented in this chapter.

\section{Effect of reference velocity}

In typical gas turbine applications, varying the loading on the engine does not affect the reference velocity.
However, since the reference velocity is a design parameter, the effect it has on the flame characteristics has implications for the design of future LSB-based gas turbine engines.

One of the key objectives of this thesis is to investigate how the LSB flame stabilization operates at high pressure conditions.
The simple model described earlier predicts a self-similar flow field for the LSB at all reference velocities.
This implies that the reference velocity will have no discernible impact on the flame standoff distance.
This result is very desirable for gas turbine designers, since the flame location and shape can be assumed to be constant.
Limited testing conducted in earlier works confirms this behavior at atmospheric pressure conditions with no preheat.

In order to verify the validity of this model at high pressure conditions in the presence of substantial preheat, the LSB was operated at a pressure of 6 atm over a range of reference velocities from 10 m/s to 40 m/s. For these tests, the \(S_{37^\circ}\) swirler was used.
In a parallel series of tests, the \(S_{45^\circ}\) swirler was tested at a pressure of 3 atm at a reference velocities of 40 and 80 m/s.
The location of the flame was measured from CH* chemiluminescence images and the results are presented in Figure FIXME.

There is essentially no systematic variation in the flame standoff distance or the flame angle for the low velocity, \(S_{37^\circ}\) tests.
The increase in reference velocity continues to produces a concomitant increase in the turbulent flame speed at the flame stabilization location, negating any change in the flame's location.
In other words, the flow field appears to retain its self-similarity, even at elevated pressures and temperatures.

However, when the \(S_{45^\circ}\) swirler was tested at higher reference velocities, the flame location shifted downstream sharply.
This indicates potential limitations to the simple flame stabilization model that may not predict the behavior of the LSB flame at elevated pressures and temperatures, particularly at high reference velocities.

To examine the probable cause of this limitation more closely, consider the effect of increasing the reference velocity on the turbulent combustion regime where the LSB combustor operates.
Previous studies have primarily operated the LSB in the flamelet regime where the modified Damk\"ohler model predicts the behavior of the turbulent flame speed with reasonable fidelity.
At elevated pressures, both the laminar flame speed of the reactants, \(S_L\) and the flame thickness, \(\delta_f\) are diminished.
This places the operating regime higher and more to the right on a Borghi diagram, as shown in Figure FIXME.
While previously, increasing the reference velocity did not affect the turbulent combustion regime, at elevated pressures, the flame is more likely to transition into the thin reaction zone.
This transition causes a drop-off in the \(S_T/S_L\) plot and the turbulent flame speed no longer increases in step with the increased levels of turbulence.
This results in the observed downstream shift of the high pressure LSB flame at high reference velocities.

\section{Effect of preheat temperature}

% Chapter 2 (background) should have presented the LDV investigation of the LSB flow field and by now, we know what it generally looks like.
% So, here, we present the axial profile at three different conditions.
% Note the similarity of the Reynolds number.
% Even if the Re is very similar, the slope is different if the temperature is higher.
% So, the effect of preheating the flow is to make it decelerate faster.
% This needs to be approached from the point of view of the dissipation rate.
% The higher preheat temperature increases the viscosity of the incoming reactants.
% Increased viscosity means the radial transport of momentum is enhanced, causing the flow to slow down faster.
% Perhaps this can also be explained by noting how a round jet might behave with increasing temperatures.

% Since the pressure section segues into the CH PLIF imaging section, we could talk about equivalence ratio and swirler angle here.
% Alternately, these two sections could come after the pressure discussion, too.

\section{Effect of swirler vane angle}

As described in Chapter FIXME 3, the LSB swirlers tested for this study are designed to have the same mass flow splits.
The \(S_{45^\circ}\) swirler has a higher vane angle, resulting in greater blockage to the flow passing through the annular section.
In order to compensate for this, the perforated plate covering the central section has slightly smaller holes.
The net effect retains the same mass flow split as in the \(S_{37^\circ}\) swirler.

Earlier, in Chapter FIXME 2, we discussed how the swirler vane angle relates to the amount of swirl imparted to the incoming flow.
According to Equation FIXME, a swirler with a higher vane angle will produce greater swirl in the reactants.
Previous work in swirl combustion\cite{} has pointed out that increased swirl shortens the flame by enhancing the swirl-induced radial pressure gradients.
The data acquired in the present investigation is in agreement with this observation.
Operated at identical inlet conditions, the \(S_{45^\circ}\) swirler stabilizes a flame closer to the dump plane and with a larger flame angle compared to the \(S_{37^\circ}\) swirler.

This result highlights an interesting trade-off for the designers of LSB-based gas turbine engines.
The \(S_{45^\circ}\) flame is located further upstream and has a more concentrated region of heat release.
This enhances the strength of the toroidal recirculation zone near the dump plane, which may be powerful enough under certain conditions (as we shall see in the Section FIXME) to even cause the flame to attach itself to the lip of the inlet.
All of this means that the \(S_{45^\circ}\) flame is more stable and will resist perturbations in the incoming flow better than the \(S_{37^\circ}\) flame.
However, the presence of a strong recirculation zone in the flow field of the \(S_{45^\circ}\) swirler will entrain more hot products and retain them longer near the zone of heat release.
This is a recipe for the production of thermal \ce{NO_x}.
While no emission measurements were made as part of this study, it may be reasonably anticipated that the \ce{NO_x} performance of the \(S_{45^\circ}\) swirler is worse than the \(S_{37^\circ}\) swirler.
The trade-off for gas turbine engine designers is thus between flame stability and emissions performance.

\section{Effect of equivalence ratio}

The LSB is primarily intended for fuel-lean operation in order to utilize its low \ce{NO_x} emission performance.
As a result, most of the testing was done as close to the target \(\phi\) of 0.56 as possible.
However, limited testing was done at 12 atm at both a slightly rich (\(\phi \approx 0.58\)) and a slightly lean (\(\phi \approx 0.53\)) condition to explore the sensitivity of the lSB flame to limited changes in equivalence ratio.
The \(S_{45^\circ}\) swirler was used for these tests.
The corresponding averaged and Abel-deconvoluted flame images are presented in Figure FIXME.

Two characteristics of the flame are immediately obvious from these images.

First, the zone of heat release, marked by the region from which CH* chemiluminescence is observed, is increasingly compact at fuel-rich conditions.
Virtually all other flame images acquired at a leaner condition show a long flame, with the heat release distributed over the entire visible area of the combustor.
The compactness of the heat release zone indicates potentially poor \ce{NO_x} performance at these conditions.

Second, the fuel-rich flame brush can be observed to wrap around and anchor itself on the dump plane.
This is particularly observable in the Abel-deconvoluted image.
The attached region is not as bright as the rest of the flame brush, indicating that the flame may be attaching itself intermittently.
This intermittent behavior can be confirmed from the instantaneous images where it is visible on some of the acquired images, but not others.
This behavior was alluded to in Section FIXME as being the result of the enhanced toroidal recirculation zone produced by this swirler.
Thus, the intermittent attachment of the flame to the inlet indicates the increased importance of the toroidal recirculation zone in stabilizing the flame.

It should be noted that the reliance on a toroidal recirculation zone to anchor the flame to the inlet is one of the primary flame stabilization mechanisms used by traditional swirl combustors.
Thus, LSB swirlers with high vane angles tend to behave like traditional swirl combustors at fuel-rich conditions.

\section{Effect of pressure}
% discuss the effect of pressure on a reacting flow field in terms of its effect on Re and on flame speed.
% Refer to previous section that may indicate that Re may not be an important parameter.
% So, we don't worry about Re and keep temperature constant and increase pressure.
% Explain behavior at low pressure ranges as being due to the temperature increase.
% At high pressure, the flame moves downstream.
% Note that the flame speed model is again insufficient to explain this.
% We can talk about local extinctions as being a possible factor, but it's actually less likely for the flame to extinguish at pressure.
% High pressure flames are thinner, less susceptible to strain effects, less likely to encounter the extinction strain rate.
% We should talk about the Borghi diagram and what happens as we move up in pressure.
% First hand, we move up and to the right, but we move to the right faster.
% If we start in the broken flamelet regime and we move into laminar flamelets, we could encounter a decrease in the flame speed.
% Then, if laminar flame speed decreases (with pressure), turbulent flame speed will fall, too.
% If we transition from broken flamelets to laminar flamelets, this could be seen in PLIF images...

\section{Flame structure}
% Focus on PLIF results only.
% The contents of this section are going to depend on what we find out.
% What does the flame do?
% What insight does that provide into the LSB flame behavior?


% Planned series of experiments to visualize the flame structure for one swirler (vane angle 40 degrees)
% Total of 10 cases
% All CH4 + air
% 300 K, 1 atm, 30 m/s, 0.70 (cold)
% 500 K, 1 atm, 30 m/s, 0.70 (baseline)
% 500 K, 1 atm, 30 m/s, 0.80 (rich)
% 500 K, 1 atm, 30 m/s, 0.60 (lean)
% 500 K, 1 atm, 20 m/s, 0.70 (slow)
% 500 K, 1 atm, 40 m/s, 0.70 (fast)
% 500 K, 1 atm, 30 m/s, 0.70, high/low turbulence
% 500 K, x atm, 30 m/s, 0.70
% Some high pressure cases would be nice.
% We'll keep it open for fuels. CH4 + H2 + air, CO + H2 + CH4 + air, perhaps.
% Cases cover:
% Effect of temperature
% Effect of equivalence ratio
% Effect of velocity
% Effect of turbulence intensity
% Effect of pressure
% Potentially, effect of fuels.


