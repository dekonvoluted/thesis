\chapter{LSB Flame Characteristics}
\label{ch:lsb}

% The primary goal is as follows:

% Investigate the behavior of the LSB at GT conditions.
%  a. Flame characteristics and behavior with velocity at high pressure conditions.
%  b. Behavior with respect to other flow and geometric parameters.
%  c. Flame structure at high pressure conditions. Requires CH PLIF.

% Key questions:
% a. At atmospheric pressure, the flame standoff distance does not vary at all with increasing velocity.
%    Is this still true when approaching closer to gas turbine operating conditions?
%    What does that say about the model used to explain this behavior?

% b. If velocity has no effect on the flame characteristics, what about other flow parameters?
%    What effect does the preheat temperature have on the flame?
%    What happens to the flame at very high pressures?
%    How does the flame look at rich and lean operating conditions?
%    Does changing the swirler have an effect on the flame?

% c. How does the flame structure look at low and high pressure conditions?
%    Does this provide any insight into the flame behavior at these conditions?
%    What physical processes may be responsible for that?

Chapter \ref{ch:background} introduced the salient features of the LSB flow field and discussed the mechanisms that enable the LSB to stabilize a flame.
Four flow parameters were introduced that sufficiently describe an operating condition of the LSB---the combustor pressure, \(p\), the preheat temperature, \(T\), the reference velocity, \(U_0\), and the equivalence ratio of the premixed reactants, \(\phi\).
The angle of the swirler vanes, \(\alpha\)---a geometric parameter---was also identified as a variable of interest.
The effect of varying these parameters on the flame characteristics constitutes the subject of this chapter.
The LSB flame was characterized by its location, its shape and its structure.
The first two of these are quantified by the flame standoff distance, \(X_f\), and the flame cone angle, \(\theta_f\) respectively.

In the same chapter, the existing theories explaining LSB operation were outlined.
These theories were developed based on observations of the LSB flame and flow field from experiments conducted at atmospheric pressure, using low velocity, non-preheated reactants.
To recapitulate one of the objective of this thesis, our goal is to reexamine the validity of these theories at conditions closer to the ones at which gas turbines may be expected to operate.

Finally, Chapter \ref{ch:background} examined key parameters that affect each of the flame characteristics, using the case of varying the reference velocity as an example.
The flame location and shape were controlled by factors affecting the axial velocity profile and the turbulent flame speed.
A Borghi diagram based on turbulence and laminar flame properties was introduced to designate the regime of premixed turbulent combustion that best describes the flame structure.

The following sections examine the results of experimental investigation of the flame characteristics conducted at high preheat and high pressure conditions.

\section{Effect of Reference Velocity}
\label{sec:lsb-effect-of-reference-velocity}

\begin{figure}

\begin{subfigure}{\linewidth}
  \centering
  \input{figures/referenceVelocityDistancePlot}
  \caption{Flame standoff distance as a function of the reference velocity}
  \label{fig:referenceVelocityDistance}
\end{subfigure}

\begin{subfigure}{\linewidth}
  \centering
  \input{figures/referenceVelocityAnglePlot}
  \caption{Flame cone angle as a function of the reference velocity}
  \label{fig:referenceVelocityAngle}
\end{subfigure}

\caption[Effect of reference velocity on the flame location and shape]{The plots show the effect of changing the reference velocity on the flame location and flame shape. The \textcolor{blue}{blue} curves are from tests conducted using the \(S_{37^\circ}\) swirler at 6 atm, while the \textcolor{red}{red} curves are from tests using the \(S_{45^\circ}\) swirler at 3 atm.}

\label{fig:referenceVelocityResults}

\end{figure}


\begin{table}
  \caption[Test conditions for Reference Velocity]{Test conditions for Reference Velocity. FIXME}
  \begin{center}
    \begin{tabular}{lcccc}
      Experiment & \(p\) & \(T\) & \(\phi\) & \(U_0\) \tabularnewline
      & atm & K & & m/s \tabularnewline
      \hline\hline
      Chemiluminescence & & & & \tabularnewline
      \hline
      & 6.04 & 525 & 0.57 \(\pm\) 0.02 & 37.3 \tabularnewline
      & 6.03 & 534 & 0.55 \(\pm\) 0.07 & 20.1 \tabularnewline
      & 6.19 & 524 & 0.55 \(\pm\) 0.07 & 18.9 \tabularnewline
      & 6.19 & 493 & 0.51 \(\pm\) 0.16 & 11.7 \tabularnewline
      & & & & \tabularnewline
      & 3.01 & 485 & 0.57 \(\pm\) 0.04 & 36.7 \tabularnewline
      & 3.02 & 523 & 0.57 \(\pm\) 0.02 & 79.5 \tabularnewline
      & & & & \tabularnewline
      CH PLIF & & & & \tabularnewline
      \hline
      & 1.0 & 315 & 0.90 & 21 \tabularnewline
      & 1.0 & 443 & 0.90 & 40 \tabularnewline
      \hline
    \end{tabular}
  \end{center}
  \label{tab:referenceVelocityCases}
\end{table}


\begin{figure}

\centering

\hfill
\begin{subfigure}{0.45\linewidth}
  \centering
  \input{figures/referenceVelocityLowVelPLIFImage}
  \caption{\(U_0\) = 21 m/s}
  \label{fig:referenceVelocityLowVelPLIFImage}
\end{subfigure}
\hfill
\begin{subfigure}{0.45\linewidth}
  \centering
  \input{figures/referenceVelocityHighVelPLIFImage}
  \caption{\(U_0\) = 40 m/s}
  \label{fig:referenceVelocityHighVelPLIFImage}
\end{subfigure}
\hfill

\hfill
\begin{subfigure}{0.45\linewidth}
  \centering
  \input{figures/referenceVelocityLowVelPLIFEdge}
  \caption{Intensity edges}
  \label{fig:referenceVelocityLowVelPLIFEdge}
\end{subfigure}
\hfill
\begin{subfigure}{0.45\linewidth}
  \centering
  \input{figures/referenceVelocityHighVelPLIFEdge}
  \caption{Intensity edges}
  \label{fig:referenceVelocityHighVelPLIFEdge}
\end{subfigure}
\hfill

\hfill
\begin{subfigure}{0.45\linewidth}
  \centering
  \input{figures/referenceVelocityLowVelPLIFHistogram}
  \caption{\(\mu\) = 364; \(\sigma\) = 167}
  \label{fig:referenceVelocityLowVelPLIFHistogram}
\end{subfigure}
\hfill
\begin{subfigure}{0.45\linewidth}
  \centering
  \input{figures/referenceVelocityHighVelPLIFHistogram}
  \caption{\(\mu\) = 463; \(\sigma\) = 187}
  \label{fig:referenceVelocityHighVelPLIFHistogram}
\end{subfigure}
\hfill

\caption[Effect of reference velocity on the flame structure]{The sequence of images on the left and the right pertain to the low and high velocity cases respectively. Each instantaneous frame of PLIF data is processed to detect edges and the statistics of the edge pixels from 300 such frames in the central quarter of the image are plotted as a histogram.}

\label{fig:referenceVelocityPLIFResults}

\end{figure}



Section \ref{subsubsec:flame-characteristics-standoff} explored how changing the reference velocity is not expected to affect the flame location or the flame shape as the balance between the local reactant velocity and the turbulent flame speed remains unaffected.
This was further borne out by Equation \ref{eqn:chengModel} which outlines a simple model for the turbulent flame speed as a linear function of \(u'\) and hence, \(U_0\).
Combined with the self-similar velocity field in the LSB, this predicted that the flame shape and location will be constant over a wide range of operating conditions.
Cheng et al.\cite{2008-cheng-a} reported no significant deviation from this model over the range of conditions at which they tested the LSB design.
It is worth repeating that these experiments were confined to low flow velocities at atmospheric pressure, non-preheated conditions.

In typical gas turbine applications, the reference velocity is not generally variable with engine loading.
Hence, the motivation for studying its effect on the flame characteristics arises from the fact that the reference velocity is a design parameter and its effect on the flame has implications for the design of future LSB-based gas turbine engines.
If the atmospheric pressure model holds at high pressure conditions and the reference velocity has no discernible effect on the flame shape and location, such behavior is desirable from the point of view of a gas turbine designer as it would simplify models for heat transfer and combustor length.

In order to verify the validity of this model at high pressure conditions in the presence of substantial preheat, the LSB was operated at a pressure of 6 atm and the reference velocity was varied from 10 m/s to 40 m/s.
For these tests, the \(S_{37^\circ}\) swirler was used.
In a parallel series of tests, the \(S_{45^\circ}\) swirler was tested at a pressure of 3 atm at reference velocities of 40 m/s and 80 m/s.
The preheat temperature for these tests was about 500 K.
The measured and calculated flow parameters for these conditions are presented in Table \ref{tab:referenceVelocityCases}.
The location of the flame was measured from CH* chemiluminescence images and the results are presented in Figure \ref{fig:referenceVelocityResults}.

There is essentially no systematic variation in the flame standoff distance or the flame angle for the low velocity, \(S_{37^\circ}\) tests.
This is in line with Equation \ref{eqn:flameImmobility}'s prediction and confirms its applicability even at elevated pressure and preheat conditions.

When the \(S_{45^\circ}\) swirler was tested at higher reference velocities, however, the flame location shifted downstream sharply.
This indicates potential limitations to the simple flame stabilization model that may not predict the behavior of the LSB flame at elevated pressures and temperatures, particularly at high reference velocities.

A possible explanation for this observation could be gleaned from considering the effect that increasing the reference velocity has on the turbulent combustion regime in which the LSB combustor operates.
Previous studies have operated the LSB in regimes where Equation \ref{eqn:chengModel} predicts the variation of the turbulent flame speed with \(u'\) with reasonably fidelity.
The arguments behind the formulation of Equation \ref{eqn:chengModel} assume that a turbulent flame can be treated as a distorted/wrinkled laminar flame in the presence of large scale, low intensity turbulence.
This assumption is largely true in the wrinkled flamelet regime and may even be extended apply to a mildly corrugated flame.
However, as we approach large \(\dfrac{ u' }{ S_L }\) values and operate in the thin reaction zone---where most gas turbine combustors operate---the flame bears little resemblance to a wrinkled laminar sheet.
Since increasing the reference velocity traverses the operating point along the vertical axis of the Borghi diagram, it can cause the turbulent combustion regime to change at high values of \(\dfrac{ u' }{ S_L }\), resulting in the ``bending effect'' in the \(\dfrac{ S_T }{ S_L }\) vs \(\dfrac{ u' }{ S_L }\) diagram.

One way to ascertain the regimes in which the LSB is operated is to image the flame sheet and observe the flame structure.
To that end, the LSB (Configuration B) was tested at atmospheric pressure with preheat temperatures ranging from 300 K to 440 K.
Two of these conditions are relevant to this discussion and their operating parameters are presented in Table \ref{tab:referenceVelocityCases}.
In order to prevent flame blow-off, the equivalence ratio had to be increased to 0.9.
The resulting flame illuminated by an 80 mm tall, 250 \(\mu\)m thick laser sheet from the alexandrite laser and imaged using the PI-MAX 512\(\times\)512 intensified camera equipped with a 50 mm, f/1.4 lens.
The camera was gated to 300 ns centered on the 70 ns laser pulse.
The laser was operated at 10 Hz and the pulse energy was measured to be about 14 mJ for the low velocity case and about 17 mJ for the high velocity case.
A sample frame from each dataset is shown in Figure \ref{fig:referenceVelocityPLIFResults}. 

Both the low and high velocity cases show an essentially unbroken flame sheet that is characterized by wrinkles.
This is consistent with the operating regime being in the laminar flamelets region of the Borghi diagram.
The level of wrinkling in these images can be estimated by gathering statistics on the number of pixels that are found on edges of intensity gradients.
In order for the statistics to be affected only by the level of wrinkling and not by the movement of the flame, the statistics are gathered only from the middle quarter of the flame where the flame sheet is nearly planar.
The edges and corresponding histograms from this investigation are also presented in Figure \ref{fig:referenceVelocityPLIFResults}.

The level of wrinkling is predictably higher for the high velocity, high preheat case.
The instances of disjoint packets of flame are also higher for this case.
This indicates that the high velocity, high preheat case is approaching the boundary between the laminar wrinkled flame regime and the corrugated flame regime.
Even accounting for the effects of operating this flame at a leaner equivalence ratio, at elevated pressures, both the laminar flame speed, \(S_L\), and the flame thickness, \(\delta_f\), are diminished.
This places the operating regime higher (and more to the right) on the Borghi diagram where a large enough reference velocity can push the operating point from the region of corrugated flames into the broken flamelets regime.
This would have not been possible for similar increases in reference velocity at atmospheric pressure conditions where the corresponding \(\dfrac{ u' }{ S_L }\) values would be too low.
Such a transition would cause \(S_T\) to increase at a reduced rate with \(u'\) and \(U_0\), resulting in the observed downstream shift of the high pressure LSB flame at high reference velocities.

\section{Effect of Preheat Temperature}
\label{sec:lsb-effect-of-preheat-temperature}

\begin{table}
  \caption[Test conditions for studying preheat temperature effects]{The following table lists the conditions at which the LSB flow field was mapped using LDV. These experiments were performed on Configuration A.}
  \begin{center}
    \begin{tabular}{lcccc}
      Case & \(T\) & \(\phi\) & \(U_0\) & \(Re\) \tabularnewline
      & K & m/s & & \tabularnewline
      \hline\hline
      & & & & \tabularnewline
      Cold, Non-reacting (CN) & 300 & - & 30 & 72700 \tabularnewline
      Hot, Non-reacting (HN) & 500 & - & 75 & 75200 \tabularnewline
      Hot, Reacting (HR) & 465 & 0.55 & 30 & 39500 \tabularnewline
      \hline
    \end{tabular}
  \end{center}
  \label{tab:temperatureCases}
\end{table}


\begin{figure}

\centering

\input{figures/temperatureLDVResultsPlot}

\caption[Effect of preheat temperature on the LSB flow field - I]{The plot shows results from LDV mapping of the flow field in the LSB. Mean axial velocity profiles are shown for \textcolor{blue}{cold}, \textcolor{green}{hot}, and \textcolor{red}{reacting} flow fields. The rms velocities are shown by dashed lines.}

\label{fig:temperatureLDVResults}

\end{figure}


\begin{figure}

\centering

\input{figures/temperatureLDVTransversePlot}

\caption[Effect of preheat temperature on the LSB flow field - II]{The plot shows LDV data from two orthogonal traverses (along \(Y\) and \(Z\) axes) performed across the LSB combustor. Mean axial velocity is presented for cold and hot cases (both non-reacting) to highlight the effect of preheat on the mean axial velocity profile through the shear layer. The traverses are performed 15 mm downstream of the inlet and use the \(S_{45^\circ}\) swirler in Configuration A.}

\label{fig:temperatureLDVTransverse}

\end{figure}



Next, we consider the effect of increasing the preheat temperature on the flame location, shape and structure.
The preheat temperature of the reactants is an important flow parameter for gas turbine applications and impacts the reacting flow field in the LSB in two ways.
First, a higher preheat temperature increases the kinematic viscosity of the flow.
The mechanism that transports the axial momentum in the radial direction and causes the flow to diverge is driven by viscous effects.
Enhanced viscous effects will result in a steeper decrease in the axial velocity.
Simultaneously, this causes an increased production of turbulent kinetic energy.
Second, global reaction rates increase with temperature, causing the laminar flame speed to increase.
The net result of the increased turbulence and laminar flame speed is that the turbulent flame propagates faster.
As a result of the steeper velocity decay and the increased turbulent flame speed, the flame will be expected to stabilize closer to the inlet of the LSB.

In order to explore this in greater detail, the velocity field of the combustor was mapped using Laser Doppler Velocimetry (LDV).
The conditions were chosen to study the effect of increasing the preheat temperature on both reacting and non-reacting LSB flow fields.
Further, the study includes both low and high reference velocity cases.
The relevant flow parameters relating to these tests are presented in Table \ref{tab:temperatureCases}.
The test cases are named CN (Cold, Non-reacting), HN (Hot, Non-reacting) and HR (Hot, Reacting) based on their preheat and presence of a flame.
All LDV tests were limited to atmospheric pressure conditions.
Implementing the LDV technique at elevated pressures proved difficult due to beam steering issues, coupled with impractical turn-around times between the successive runs that would be required to obtain sufficient LDV data points for analysis.

The normalized centerline mean and rms axial velocity profiles for the three cases are presented in Figure \ref{fig:temperatureLDVResults}.
The abscissa represents the distance from the virtual origin as defined in Section \ref{subsec:lsb-axial-velocity-profile}.

The results show that increasing the preheat temperature causes the normalized velocity slope to increase.
As noted in Chapter \ref{ch:background}, Cheng et al.\cite{2008-cheng-a} reported that this slope is unaffected by the Reynolds number, \(Re\), of the operating condition.
The results in Figure \ref{fig:temperatureLDVResults} however, show that even though the cases CN and HN have similar \(Re\), their mean velocity profiles have very different slopes.
This indicates that the mean axial stretch in the near field of the LSB flow field is a stronger function of the preheat temperature than \(Re\).

Even assuming that \(S_T\) is constant, these results suggest that at higher preheat temperatures, the flame would stabilize closer to the dump plane because of the faster reactant velocity decay.
In fact, the faster velocity decay produces a higher \(u'\) and would increase \(S_T\), further causing the flame standoff location to shift upstream.
This affects the stability of the flame in two ways.
First, the flame is closer to the inlet, resulting in more effective heat transfer from the flame zone into the reactants.
Second, the steeper profiles of \(U\) and \(S_T\) limit any movement of the flame in response to perturbations in the local flow field.
These two characteristics of the preheated flame lead to an intuitive result---the LSB flame behaves more stably at high preheat conditions.

The effect of increasing the preheat temperature on the flame cone angle is harder to predict.
The increased divergence drives more axial momentum into the shear layer, increasing the local velocity of the reactants.
At the same time, the increased \(u'\) and \(S_L\) drive up the turbulent flame speed.
The competing effects of the two parameters on the flame angle can be explored by propagating a change in the parameters as follows.

\begin{align}
  U \sin \theta_f &= S_T \nonumber \\
  \therefore \frac{ \delta \sin \theta_f }{ \sin \theta_f } &= \frac{ \delta S_T }{ S_T } - \frac{ \delta U }{ U }
\end{align}

Figure \ref{fig:temperatureLDVTransverse} shows LDV results for the axial velocity measured 15 mm downstream of the inlet.
These measurements are taken from two orthogonal traverses across the flow field and show the profiles for the CN and HN cases.
While the increased momentum transport has certainly thickened the shear layer, the peak axial velocity remains the same for the two cases.
As a result, changes in the turbulent flame speed would be expected to dominate and cause the flame angle to increase.

Finally, we return to the Borghi diagram to consider the effect of a higher preheat temperature on the flame structure.
On the Y-axis, the exponential dependence of \(S_L\) on temperature will decrease the ordinate of the operating point, while on the X-axis, a slight decrease in the flame thickness will move the point to the right.
The net down-and-right movement of the operating point drives the flame regime away from broken flamelets and closer towards wrinkled flames.
Since our flame structure data was already in the wrinkled flame regime, this change could not be imaged by CH PLIF.

\section{Effect of Swirler Vane Angle}
\label{sec:lsb-effect-of-swirler-vane-angle}

As described in Chapter \ref{ch:experimental}, the LSB swirlers tested for this study are designed to have the same mass flow splits.
The \(S_{45^\circ}\) swirler has a higher vane angle, resulting in greater blockage to the flow passing through the annular section.
In order to compensate for this, the perforated plate covering the central section has slightly smaller holes.
The net effect retains the same mass flow split as in the \(S_{37^\circ}\) swirler.

Chapter \ref{ch:background} describes how the swirler vane angle relates to the amount of swirl imparted to the incoming flow.
According to Equation FIXME, a swirler with a higher vane angle will produce greater swirl in the reactants.
% DOES THAT EXPLANATION DESCRIBE THERE IS NO SWIRL IN THE CENTRAL REGION - IN WHICH CASE YOU MIGHT NOT WANT TO SAY HOW IT EFFECTS THE SWIRL OF THE REACTANTS - BUT INSTEAD THE SWIRL OF THE OUTER REACTANT FLOW?
Previous work in swirl combustion\cite{1992-chan,1986-starner} has pointed out that increased swirl shortens the flame by enhancing the swirl-induced radial pressure gradients.
% YOU WILL NEED TO EXPLAIN HOW THIS IS RELEVANT HERE - AS THE CENTRAL REGION IS NO SWIRL?
The data acquired in the present investigation is in agreement with this observation.
Operated at identical inlet conditions, the \(S_{45^\circ}\) swirler stabilizes a flame closer to the dump plane and with a larger flame angle compared to the \(S_{37^\circ}\) swirler.

This result highlights an interesting trade-off for the designers of LSB-based gas turbine engines.
The \(S_{45^\circ}\) flame is located further upstream and has a more concentrated region of heat release.
This enhances the strength of the toroidal recirculation zone near the dump plane, which may be powerful enough under certain conditions (as we shall see in Section \ref{sec:lsb-effect-of-equivalence-ratio}) to even cause the flame to attach itself to the lip of the inlet.
All of this means that the \(S_{45^\circ}\) flame is more stable and will resist perturbations in the incoming flow better than the \(S_{37^\circ}\) flame.
However, the presence of a strong recirculation zone in the flow field of the \(S_{45^\circ}\) swirler will entrain more hot products and retain them longer near the zone of heat release.
This is a recipe for the production of thermal \ce{NO_x}.
While no emission measurements were made as part of this study, it may be reasonably anticipated that the \ce{NO_x} performance of the \(S_{45^\circ}\) swirler will be degraded compared to the \(S_{37^\circ}\) swirler.
The trade-off for gas turbine engine designers is thus between flame stability and emissions performance.

% Text dump from Chapter 2
The swirler vane angle directly affects the amount of swirl present in the flow field.
Since the LSB utilizes swirl to enhance the divergence of the flow near the inlet, an increased vane angle will result in a sharper deceleration of the reactants and a more upstream stabilization location of the turbulent flame.
This is corroborated by past research\cite{1986-starner} which reported shorter, wider (higher flame angle) flames when the amount of swirl in a combustor was increased.
Increasing the swirl does not affect any other parameters in the flow and is hence not expected to have a significant effect on the flame structure.

\section{Effect of Equivalence Ratio}
\label{sec:lsb-effect-of-equivalence-ratio}

The LSB is primarily intended for fuel-lean operation in order to utilize its low \ce{NO_x} emission performance.
As a result, most of the testing was done as close as possible to a target \(\phi\) of 0.56.
Limited testing was carried out at 12 atm for two off-target conditions: a slightly richer (\(\phi \approx 0.58\)) and a slightly leaner (\(\phi \approx 0.53\)) mixture, in order to explore the sensitivity of the LSB flame to limited changes in equivalence ratio.
The \(S_{45^\circ}\) swirler was used for these tests.
The corresponding averaged and Abel-deconvoluted flame images are presented in Figure FIXME.

Two characteristics of the flame are immediately obvious from these images.
First, the zone of heat release, marked by the region from which CH* chemiluminescence is observed, is increasingly compact at fuel-rich conditions.
Virtually all other flame images acquired at a leaner condition show a long flame, with the heat release distributed over the entire visible area of the combustor.
The compactness of the heat release zone indicates potentially poor \ce{NO_x} performance at these conditions.
% HMMM - SLIGHTLY RICHER WOULD ALWAYS RAISE NOX - YOU NEED TO EXPLAIN WHAT YOU MEAN HERE - IF YOU ARE TALKING ABOUT LONGER HIGH T RESIDENCE TIMES, WHAT LEVEL OF FRACTIONAL INCREASE ARE WE TALKING ABOUT??

Second, the fuel-rich flame brush can be observed to wrap around and anchor itself on the dump plane.
This is particularly observable in the Abel-deconvoluted image.
The attached region is not as bright as the rest of the flame brush, indicating that the flame may be attaching itself intermittently.
This intermittent behavior can be confirmed from the instantaneous images where it is visible on some of the acquired images, but not others.
% ARE YOU GOING TO SHOW THEM??
This behavior was alluded to in Section \ref{sec:lsb-effect-of-swirler-vane-angle} as being the result of the enhanced toroidal recirculation zone produced by this swirler.
Thus, the intermittent attachment of the flame to the inlet indicates the increased importance of the toroidal recirculation zone in stabilizing the flame.

It should be noted that the reliance on a toroidal recirculation zone to anchor the flame to the inlet is one of the primary flame stabilization mechanisms used by traditional swirl combustors.
Thus, LSB swirlers with high vane angles tend to behave like traditional swirl combustors at fuel-rich conditions.

% Text dump from Chapter 2
Operating the LSB with a richer (or more specifically, a less lean) fuel-air mixture can significantly increase the net heat release in the combustor .
While the velocity flow field is unaffected by changing the reactant composition (except in response to the heat release), the turbulent flame speed is enhanced.
Thus, increasing the equivalence ratio will cause the flame to move upstream and stabilize closer to the inlet.
The response of the flame angle to this change is not immediately discernible.

On the Borghi diagram, the increase in \(S_L\) and decrease in flame thickness caused by the richer mixture will drive the operating point down and to the right, towards the wrinkled flamelets regime.

\section{Effect of Combustor Pressure}
\label{sec:lsb-effect-of-combustor-pressure}

In many gas turbine engines, the combustor pressure varies directly with the loading of the engine.
Like the preheat temperature, the combustor pressure affects the LSB flame both through the fluid mechanics of the flow and the kinetics of the chemical reactions in the flame.
The effect of the combustor pressure on the fluid mechanics of the LSB flow field can be captured by its effect on the Reynolds number.
As noted in Section \ref{sec:lsb-effect-of-preheat-temperature}, however, previous work indicated the Reynolds number may not be an important parameter for the LSB, particularly in the near field where the flame stabilization occurs. 
On the other hand, the effect of the combustor pressure on reaction rates in the flame is clearly important.
Increasing the combustor pressure results in a lower laminar flame speed and reduced flame thickness for methane-air flames.
According to the modified Damk\"ohler model discussed earlier, the reduced laminar flame speed should have little or no effect on the flow field, since the contribution from \(S_L\) in Equation FIXME is vanishingly small, even at the lowest reference velocities of our test conditions.
However, as suggested by our discussion in Section \ref{sec:lsb-effect-of-reference-velocity}, the validity of the simple model at elevated pressure conditions is questionable.

In order to resolve the uncertainties regarding how the LSB flame responds to combustor pressure, the flame was imaged over a range of operating conditions from 3 to 12 atm.
For these tests, the reference velocity and the equivalence ratio were held constant.
However, the temperature of the reactants continues to increase with pressure.
The reason for this was discussed in Chapter \ref{ch:experimental} and is attributable to the reduced heat losses in the connecting pipes at the high flow rates required to pressurize the LSB.
The flame location and shape inferred from the flame images are presented in Figure FIXME.

At low to moderate pressures, the flame location is nearly invariant for \(S_{37^\circ}\), but moves upstream for the \(S_{45^\circ}\) cases.
This behavior can be explained as follows.
The flame stabilization location for the \(S_{45^\circ}\) swirler is closer to the dump plane compared to the \(S_{37^\circ}\) swirler.
This should result in enhanced heat transfer to the dump plane and consequently to the incoming reactants.
This feedback is even more effective as the temperature of the incoming reactants increases.
% WHY?? - EXPLAIN!!
This causes the upstream shift of the \(S_{45^\circ}\) flame, while the \(S_{37^\circ}\) flame is less affected by these processes.

At high pressures, however, both flames are observed to move downstream, despite the increasing preheat temperatures.
The apparent decrease in the turbulent flame speed at these conditions is an unexpected result, and the modified Damk\"ohler model is insufficient in accounting for this observation.
Figure FIXME also shows that the flame angle for both cases decreases slightly with pressure.
This suggests that the turbulent flame speed was consistently decreasing with pressure.
In light of this, the nearly constant location of the \(S_{37^\circ}\) flame could be attributed to the effects of increasing combustor pressure and preheat temperature nearly canceling each other out at the lower pressures.

% Text dump from Chapter 2
Finally, the effect of increasing the combustor pressure on the flame characteristics is of much pertinence it is likely to be varied in gas turbine engines to change the loading/power output.
Researchers\cite{1998-kobayashi,2002-kobayashi-b} have observed that the effect of the combustor pressure on the turbulent flame speed is minimal, causing a slight drop in the turbulent flame speed at high pressures.
Griebel et al.\cite{2007-griebel} attribute this behavior to the fact that the decrease in \(S_L\) with pressure is compensated for by broadening of the turbulence spectrum, resulting in a higher turbulent flame surface area.
Higher pressures also decrease the kinematic viscosity of the reactant flow, causes a somewhat less efficient divergence at the inlet of the LSB.
The net effect of the two is expected to cause the flame to stabilize slightly further downstream from the inlet of the LSB.

This has repercussions for the gas turbine designers as a flame stabilized further away from the inlet is likelier to be blown off due to perturbations.
In practice, operating at high combustor pressure also causes increased preheat temperatures and the two effects should counteract each other.
The slight decrease in the turbulent flame speed should cause the flame angle to decrease by a small amount.

Finally, it is interesting to examine this on a Borghi diagram to deduce any changes in the flame structure that might result.
Kobayashi et al.\cite{1997-kobayashi} report that the increasing pressure causes a weak drop in \(u'\), reaching a minimum around 10 atm.
This drop, though, is expected to be dominated by the fall in laminar flame speed with pressure, causing the \(u'/S_L\) factor to increase with pressure.
On the X-axis, the integral length scale is very nearly unaffected by the pressure change, but the flame thickness decreases rapidly.
Thus, pressure causes the operating point to move up and to the right on the diagram.
At most, this can cause the turbulent regime to change from wrinkled laminar flamelets to corrugated flames.
In the absence of a drastic change in the turbulent flame, the only observable effect is likely to be an increase in the fine structure of the wrinkles in the flame due to the broadened turbulence spectrum.

% Some sort of concluding remarks or summary of results should go here

