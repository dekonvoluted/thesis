\chapter{LSB Flame Characteristics}

% We've already talked about what the LSB is and how its flow field looks like in Chapter 2.
% Now, we discuss results from LSB flame imaging experiments.

% The primary goal is as follows:

% Investigate the behavior of the LSB at GT conditions.
%  a. Flame characteristics and behavior with velocity at high pressure conditions.
%  b. Behavior with respect to other flow and geometric parameters.
%  c. Flame structure at high pressure conditions. Requires CH PLIF.

% Key questions:
% a. At atmospheric pressure, the flame standoff distance does not vary at all with increasing velocity.
%    Is this still true when approaching closer to gas turbine operating conditions?
%    What does that say about the model used to explain this behavior?

% b. If velocity has no effect on the flame characteristics, what about other flow parameters?
%    What effect does the preheat temperature have on the flame?
%    What happens to the flame at very high pressures?
%    How does the flame look at rich and lean operating conditions?
%    Does changing the swirler have an effect on the flame?

% c. How does the flame structure look at low and high pressure conditions?
%    Does this provide any insight into the flame behavior at these conditions?
%    What physical processes may be responsible for that?

In Chapter FIXME, we introduced the salient features of the Low Swirl Burner (LSB) flow field and discussed the mechanisms by which the LSB flame is stabilized.
Further, various characteristics of the LSB flame that can be measured from flame images were outlined.
To recapitulate, these are the flame location, flame shape and the flame structure.
The first two are quantified by the flame standoff distance, \(X_f\) and the flame angle, \(\theta_f\), respectively.

In the same chapter, we introduced the four flow parameters that describe an operating condition for the LSB --- the combustor pressure, \(p\), the preheat temperature, \(T\), the mass-averaged inlet velocity (also called the reference velocity, \(U_0\) and the equivalence ratio of the premixed reactants, \(\phi\).
We further introduced a geometric parameter --- the angle of the vanes of the swirler, \(\alpha\), which affects the amount of swirl present in the flow field.

The LSB flame is imaged over a range of operating conditions and the effect of flow and geometric parameters on the reacting flow field is investigated.
This results of the investigation are presented in this chapter.

\section{Effect of reference velocity}

Generally, in gas turbine applications, the reference velocity is not expected to be varied with loading.
However, its effect on the flame characteristics has implications for the design of future LSB-based gas turbines.

% The first question relates to how the flame stabilization model works at high pressure conditions.
% So, we need to talk about the effect of velocity on the flame standoff location.
% There is little effect at low velocities.
% This seems to be in agreement with the model.
% At high velocities, there may be a discernible effect.
% This implies that the earlier model linking laminar flame speed to the turbulent flame speed may be too simplistic.
% The model derives from assuming that turbulent flames are simply laminar flames that are folded.
% We may need to talk about the Borghi diagram and talk about what regions this assumption may be valid in.
% Perhaps as we increase the velocity, keeping everything else constant, we move into a region where this model is not necessarily true.
% On the Borghi diagram, we should move up, by the way for increasing reference velocity.
% Note that the flame speed model is insufficient to explain this behavior.

\section{Effect of preheat temperature}
% Chapter 2 (background) should have presented the LDV investigation of the LSB flow field and by now, we know what it generally looks like.
% So, here, we present the axial profile at three different conditions.
% Note the similarity of the Reynolds number.
% Even if the Re is very similar, the slope is different if the temperature is higher.
% So, the effect of preheating the flow is to make it decelerate faster.
% This needs to be approached from the point of view of the dissipation rate.
% The higher preheat temperature increases the viscosity of the incoming reactants.
% Increased viscosity means the radial transport of momentum is enhanced, causing the flow to slow down faster.
% Perhaps this can also be explained by noting how a round jet might behave with increasing temperatures.

% Since the pressure section segues into the CH PLIF imaging section, we could talk about equivalence ratio and swirler angle here.
% Alternately, these two sections could come after the pressure discussion, too.
\section{Effect of equivalence ratio}
% Rich flames are shorter, more compact.
% Lean flames are longer and more diffuse.
% Talk about how chemiluminescence tracks heat release rate.

\section{Effect of swirler vane angle}
% As expected, based on literature, larger swirler vane angles means more swirl.
% Flame angles are wider.
% Flame is located closer to dump plane.
% All this is expected based on literature\cite{starner}, etc.

\section{Effect of pressure}
% discuss the effect of pressure on a reacting flow field in terms of its effect on Re and on flame speed.
% Refer to previous section that may indicate that Re may not be an important parameter.
% So, we don't worry about Re and keep temperature constant and increase pressure.
% Explain behavior at low pressure ranges as being due to the temperature increase.
% At high pressure, the flame moves downstream.
% Note that the flame speed model is again insufficient to explain this.
% We can talk about local extinctions as being a possible factor, but it's actually less likely for the flame to extinguish at pressure.
% High pressure flames are thinner, less susceptible to strain effects, less likely to encounter the extinction strain rate.
% We should talk about the Borghi diagram and what happens as we move up in pressure.
% First hand, we move up and to the right, but we move to the right faster.
% If we start in the broken flamelet regime and we move into laminar flamelets, we could encounter a decrease in the flame speed.
% Then, if laminar flame speed decreases (with pressure), turbulent flame speed will fall, too.
% If we transition from broken flamelets to laminar flamelets, this could be seen in PLIF images...

\section{Flame structure}
% Focus on PLIF results only.
% The contents of this section are going to depend on what we find out.
% What does the flame do?
% What insight does that provide into the LSB flame behavior?


% Planned series of experiments to visualize the flame structure for one swirler (vane angle 40 degrees)
% Total of 10 cases
% All CH4 + air
% 300 K, 1 atm, 30 m/s, 0.70 (cold)
% 500 K, 1 atm, 30 m/s, 0.70 (baseline)
% 500 K, 1 atm, 30 m/s, 0.80 (rich)
% 500 K, 1 atm, 30 m/s, 0.60 (lean)
% 500 K, 1 atm, 20 m/s, 0.70 (slow)
% 500 K, 1 atm, 40 m/s, 0.70 (fast)
% 500 K, 1 atm, 30 m/s, 0.70, high/low turbulence
% 500 K, x atm, 30 m/s, 0.70
% Some high pressure cases would be nice.
% We'll keep it open for fuels. CH4 + H2 + air, CO + H2 + CH4 + air, perhaps.
% Cases cover:
% Effect of temperature
% Effect of equivalence ratio
% Effect of velocity
% Effect of turbulence intensity
% Effect of pressure
% Potentially, effect of fuels.


