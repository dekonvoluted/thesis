\chapter{Background}

\section{Linearity}

In the weak excitation limit, the signal is a function of CH concentration and the rate of collisional quenching of the excited CH radicals.
In the strong excitation limit, the signal depends only on the CH concentration is unaffected by the quenching of the excited CH species.

It is difficult to ensure that the CH system is saturated spatially, temporally and spectrally at the same time.
Further, operating with high laser intensities may bleach the energy levels being excited by inducing chemical reactions that destroy the excited CH radicals.
Hence, it is preferred to operate in the linear regime.

\section{CH PLIF Quenching Model}
\label{sec:quenchingmodel}

In order to calculate the intensity of the quenched CH PLIF signal in a flame, an improved model of the CH system was constructed and analyzed.
According to this new model, CH radicals from the \(X\) ground state are excited to the \(B(0)\) upper state.
This is followed by collisional transfer to the \(A(1)\) and \(A(0)\) states.
The transfer between the nearly degenerate \(A(1)\) and \(B(0)\) states is partially reversible.
The transfer between \(B(0)\) and \(A(0)\) is not reversible.
This is followed by spontaneous emission as CH radicals transition from the \(A\) states to the \(X\) state.
This results in a pseudo-three-level model as shown in Figure FIXME.

Figure FIXME indicates the rates of the various processes discussed.
The subscripts 0, 1 and 2 represent the electronic energy levels \(X\), \(A\) and \(B\) respectively.
Processes involving the \(A(0)\) state are differentiated from those involving the \(A(1)\) state by a prime (').
With the exception of the nearly degenerate \(A(1)\) and \(B(0)\) states, most collisional excitation steps are neglected due to their low probability.

In this formulation, the signal intensity of the CH PLIF emission is given by Equation \ref{eqn:signalIntensity}.

\begin{equation}
S = ( n_1A_{10} + n'_1A'_{10} + n_2A_{20} )V
\label{eqn:signalIntensity}
\end{equation}

The spontaneous emission coefficients, \(A_{10}\), \(A'_{10}\) and \(A_{20}\) are obtained from various published papers\cite{1985-garland-a,1996-luque-b,2005-richmond}.
The values used for this analysis are presented in Table \ref{tab:emissionCoefficients}.

\begin{table}
  \caption[Einstein A coefficients]{The coefficients of spontaneous emission for transitions in the CH system are provided.}
  \begin{center}
    \begin{tabular}{lcr}
      Transition & Symbol & A, s\(^{-1}\) \tabularnewline
      \hline\hline
      \(B\rightarrow X(0,0)\) & \(A_{20}\) & \(2.963 \times 10^6\) \tabularnewline
      \(A\rightarrow X(1,1)\) & \(A_{10}\) & \(1.676 \times 10^6\) \tabularnewline
      \(A\rightarrow X(0,0)\) & \(A'_{10}\) & \(1.832 \times 10^6\) \tabularnewline
    \end{tabular}
  \end{center}
  \label{tab:emissionCoefficients}
\end{table}


Equations \ref{eqn:rates1}--\ref{eqn:rates3} describe the time variation of the number density of CH radicals in each excited state.

\begin{align}
\frac{dn_1}{dt} &= -( A_{10} + Q_{10} + R_{12} )n_1 + R_{21}n_2
\label{eqn:rates1}\\
\frac{dn'_1}{dt} &= -( A'_{10} + Q'_{10} )n'_1 + R'_{21}n_2
\label{eqn:rates2}\\
\frac{dn_2}{dt} &= W_{02}n_0 + R_{12}n_1 - ( A_{20} + Q_{20} + R_{21} + R'_{21} )n_2
\label{eqn:rates3}
\end{align}

At steady state, the rate of change of the number density is minimal.
Under this assumption, the LHS of Equations \ref{eqn:rates1}--\ref{eqn:rates3} can be set to zero.
This results in a closed set of linear equations in terms of the populations of the upper states.
This set of equations is presented in Equation \ref{eqn:closedForm}.

\begin{equation}
  \left[
    \begin{matrix}
      A_{10} + Q_{10} + R_{12} & 0 & -R_{21}\\
      0 & A'_{10} + Q'_{10} & -R'_{21}\\
      -R_{12} & 0 & A_{20} + Q_{20} + R_{21} + R'_{21}
    \end{matrix}
  \right]\left[
    \begin{matrix}
      n_1\\
      n'_1\\
      n_2
    \end{matrix}
  \right] = \left[
    \begin{matrix}
      0\\
      0\\
      W_{02}n_0
    \end{matrix}
  \right]
  \label{eqn:closedForm}
\end{equation}

The solution to Equation \ref{eqn:closedForm} is shown in Equations \ref{eqn:solution1}--\ref{eqn:solution3}.

\begin{align}
  n_1 &= \frac{ R_{21} }{ ( A_{10} + Q_{10} + R_{12} )( A_{20} + Q_{20} + R_{21} + R'_{21} ) - R_{12}R_{21} }W_{02}n_0
  \label{eqn:solution1}\\
  n'_1 &= \frac{ ( A_{10} + Q_{10} + R_{12} )R'_{21} }{ ( A'_{10} + Q'_{10} ) ( ( A_{10} + Q_{10} + R_{12} )( A_{20} + Q_{20} + R_{21} + R'_{21} ) - R_{12}R_{21} ) }W_{02}n_0
  \label{eqn:solution2}\\
  n_2 &= \frac{ ( A_{10} + Q_{10} + R_{12} ) }{ ( A_{10} + Q_{10} + R_{12} )( A_{20} + Q_{20} + R_{21} + R'_{21} ) - R_{12}R_{21} }W_{02}n_0
  \label{eqn:solution3}
\end{align}

These expressions can be further simplified by noting various observations made in studies of the CH system.
For instance, previous work\cite{1984-cool,1985-garland-b} has reported that the \(B\) state is slightly (about 1.3 times) more prone to quenching compared to the \(A\) state.
We can thus make the following assumptions.

\begin{gather}
  Q_{10} = Q'_{10} = Q
  \label{eqn:quenchingAssumption1}\\
  Q_{20} = 1.3Q
  \label{eqn:quenchingAssumption2}
\end{gather}

Next, it has been reported\cite{2000-luque} that the electronic energy transfer rate from \(B\) to \(A\) state accounts for 0.24 times the total collisional removal from the \(B\) state.

\begin{gather}
  \frac{ R_{21} + R'_{21} - R_{12} }{ Q_{20} + R_{21} + R'_{21} - R_{12} } = 0.24\\
  \therefore \frac{ R_{21} + R'_{21} - R_{12} }{ Q } = 0.4105
  \label{eqn:REquation1}
\end{gather}

We further know\cite{1985-garland-b, 2000-luque} that the collisional transfer from the \(B(0)\) energy level populates the nearly degenerate \(A(1)\) level about four times faster than the \(A(0)\) level.

\begin{equation}
  \frac{ R_{21} - R_{12} }{ R'_{21} } = 4
  \label{eqn:REquation2}
\end{equation}

Finally, it was observed\cite{1985-garland-b} that the rate of forward transfer from \(B(0)\) to \(A(1)\) is about 1.6 times the reverse process.

\begin{equation}
  \frac{R_{21}}{R_{12}} = 1.6
  \label{eqn:REquation3}
\end{equation}

Collating Equations \ref{eqn:REquation1}--\ref{eqn:REquation3}, we obtain a closed set of linear equations.
This can be solved to eliminate \(R_{21}\), \(R_{12}\) and \(R'_{21}\) in terms of \(Q\) as shown in Equation \ref{eqn:RSolution}.

\begin{equation}
  \left[
    \begin{matrix}
      R_{21}\\
      R'_{21}\\
      R_{12}
    \end{matrix}
  \right] = \left[
    \begin{matrix}
      5.1966\\
      0.4872\\
      3.2479
    \end{matrix}
  \right] Q
  \label{eqn:RSolution}
\end{equation}

Substituting Equations \ref{eqn:quenchingAssumption1}, \ref{eqn:quenchingAssumption2} and \ref{eqn:RSolution} into Equations \ref{eqn:solution1}--\ref{eqn:solution2} leads to simplified expressions for the populations of the upper electronic states purely as a function of the respective Einstein coefficients and the collisional quenching rate.
These are presented in the following Equations \ref{eqn:simplifiedSolution1}--\ref{eqn:simplifiedSolution3}.

\begin{align}
  n_1 &= \frac{ 5.1966Q }{ ( A_{10} + 4.2479Q )( A_{20} + 6.9838Q ) - 16.8780Q } W_{02}n_0
  \label{eqn:simplifiedSolution1}\\
  n'_1 &= \frac{ 0.4872Q( A_{10} + 4.2479Q ) }{ ( A'_{10} + Q ) \left( ( A_{10} + 4.2479Q )( A_{20} + 6.9838Q ) - 16.8780Q \right) } W_{02}n_0
  \label{eqn:simplifiedSolution2}\\
  n_2 &= \frac{ ( A_{10} + 4.2479Q ) }{ ( A_{10} + 4.2479Q )( A_{20} + 6.9838Q ) - 16.8780Q } W_{02}n_0
  \label{eqn:simplifiedSolution3}
\end{align}

The quenching rate, \(Q\) of excited CH radicals is calculated by using the quenching cross-sections of various species.
The quenching cross-sections are measures of the effectiveness of each collision between a given species and an excited CH radical.
The effectiveness of the collision also depends on the velocity of collision between the two species, \(g_j\) and the abundance of the species, \(n_j\).
This relationship is formalized in Equation \ref{eqn:quenchingRate}.

\begin{align}
  Q & =\sum_j g_j \sigma_j n_j \nonumber \\
  Q & = \sum_j \sqrt{\frac{ 8kT }{ \pi\mu_j }} \sigma_j \frac{ pN_A }{ RT } X_j
  \label{eqn:quenchingRate}
\end{align}

In Equation \ref{eqn:quenchingRate}, \(\mu_j\) represents the reduced mass of the colliding CH-\(j\) molecules,\(p\) is the pressure, \(N_A\) is Avogadro's Number, \(R\) is the Universal Gas Constant, \(T\) is the temperature, and \(X_j\) is the mole fraction of species \(j\).
The mole fractions of the various species in the flame, as well as the temperature across the flame are obtained from Chemkin simulations.
The expression for the reduced mass is given in Equation \ref{eqn:reducedMass}.

\begin{equation}
  \mu_j = \frac{ m_j m_{CH} }{ m_j + m_{CH} }
  \label{eqn:reducedMass}
\end{equation}

The quenching cross-sections of various species are obtained from various published papers\cite{1994-chen,1998-tamura,2002-renfro} and are functions of temperature.
The functional forms used in this study are presented in Table \ref{tab:quenchingCrossSections}.

\begin{table}
  \caption[Quenching Cross-sections]{The functional form of the quenching cross-sections of various species with CH are provided.}
  \begin{center}
    \begin{tabular}{lr}
      Species & \(\sigma\), \AA\(^2\) \tabularnewline
      \hline\hline
      \ce{H2} & \(6.1 \exp{ \left(-686 / T \right)}\) \tabularnewline
      \ce{H} & \(221 T^{-0.5} \exp{ \left( -686 / T \right)}\) \tabularnewline
      \ce{O2} & \(8.61 \times 10^{-6} T^{1.64} \exp{ \left( 867 / T \right)}\) \tabularnewline
      \ce{OH} & \(221 T^{-0.5} \exp{ \left( -686 / T \right)}\) \tabularnewline
      \ce{H2O} & 9.6 \tabularnewline
      \ce{CH4} & \(52.8 T^{-0.5} \exp{ \left( -84 / T \right)}\) \tabularnewline
      \ce{CO} & 8.31 \tabularnewline
      \ce{CO2} & \(8.67 \times 10^{-13} T^{3.8} \exp{ \left( 854 / T \right)}\) \tabularnewline
      \ce{C2H6} & 13.4 \tabularnewline
      \ce{N2} & \(1.53 \times 10^{-4} T^{1.23} \exp{ \left( -522.1 / T \right)}\) \tabularnewline
      \ce{C3H8} & 22 \tabularnewline
      \hline
    \end{tabular}
  \end{center}
  \label{tab:quenchingCrossSections}
\end{table}



The term \(W_{02}n_0\) in Equations \ref{eqn:simplifiedSolution1}--\ref{eqn:simplifiedSolution3} represents the rate of pumping of the ground state CH radicals.
The current excitation scheme targets multiple transitions in the R-bandhead.
The pumping rate for each transition is the product of the number of CH radicals present in the appropriate level, the Einstein absorption coefficient for that energy level, \(B_i\) and the amount of laser energy available at the appropriate frequency, \(E_i\).
As a result, the term is actually a summation over the individual energy levels.
Equation \ref{eqn:pumpingRate} presents this symbolically.

\begin{align}
  W_{02}n_0 &= \sum_i B_i I_i n_i \nonumber \\
  W_{02}n_0 &= \sum_i B_i \frac{ E_i }{ A_c } \frac{ pN_A X_{CH} }{ RT } f_i
  \label{eqn:pumpingRate}
\end{align}

Table \ref{tab:absorptionCoefficients} presents the values of \(B_i\) for the transitions targeted by the current excitation scheme.\cite{1996-luque-c}
Assuming a Gaussian line shape for the laser, and using the line strengths from LIFBASE, the relative amount of energy absorbed by each transition can be calculated.
These values are also presented in Table \ref{tab:absorptionCoefficients}.

\begin{table}
  \caption[Einstein B coefficients]{The coefficients of absorption for selected transitions in the CH \(X(v=0)\) system are provided.}
  \begin{center}
    \begin{tabular}{lrrr}
      \(N''\) & \(\lambda\), nm & \(B\), m\(^2\)J\(^{-1}\)s\(^{-1}\) & \(E\) (normalized) \tabularnewline
      \hline\hline
      R1 & & & \tabularnewline
      5 & 387.2698 & \(7.677 \times 10^9\) & 0.0568 \tabularnewline
      6 & 387.1899 & \(7.665 \times 10^9\) & 0.1706 \tabularnewline
      7 & 387.1677 & \(7.610 \times 10^9\) & 0.1483 \tabularnewline
      8 & 387.206  & \(7.519 \times 10^9\) & 0.1479 \tabularnewline
      9 & 387.308  & \(7.397 \times 10^9\) & 0.0126 \tabularnewline
      R2 & & & \tabularnewline
      5 & 387.2289 & \(7.539 \times 10^9\) & 0.1080 \tabularnewline
      6 & 387.1549 & \(7.569 \times 10^9\) & 0.1128 \tabularnewline
      7 & 387.1371 & \(7.539 \times 10^9\) & 0.0841 \tabularnewline
      8 & 387.1786 & \(7.464 \times 10^9\) & 0.1311 \tabularnewline
      9 & 387.283  & \(7.354 \times 10^9\) & 0.0279 \tabularnewline
    \end{tabular}
  \end{center}
  \label{tab:absorptionCoefficients}
\end{table}


In Equation \ref{eqn:pumpingRate}, \(A_c\) is the area of cross-section of the laser beam and \(f_i\) is the Boltzmann fraction of the population at the energy level \(i\).
The expression for the Boltzmann fraction at the energy level corresponding to the vibrational quantum number \(v\) and rotational quantum number \(J\) is given in Equation \ref{eqn:BoltzmannFraction}.

\begin{equation}
  f(v,J) = \frac{ \exp{\left(\dfrac{-hcE_v(v)}{kT}\right)} (2J + 1)\exp{\left(\dfrac{-hcE_r(v, J)}{kT}\right)} }{ Q_{rv} }
  \label{eqn:BoltzmannFraction}
\end{equation}

The vibrational energy, \(E_v(v)\) of a level is calculated according to Equation \ref{eqn:vibrationalEnergy}, while the rotational energy, \(E_r(v, J)\) is calculated according to Equation \ref{eqn:rotationalEnergy}.

\begin{align}
  E_v(v) &= \omega_e \left(v+\frac{1}{2}\right) - \omega_ex_e \left(v+\frac{1}{2}\right)^2 + \omega_ey_e \left(v+\frac{1}{2}\right)^3 - \omega_ez_e \left(v+\frac{1}{2}\right)^4
  \label{eqn:vibrationalEnergy}\\
  E_r(v, J) &= \left\{B_e - \alpha_e \left(v+\frac{1}{2}\right)\right\}J(J+1) - \left\{D_e + \beta_e \left(v+\frac{1}{2}\right)\right\}J^2(J+1)^2
  \label{eqn:rotationalEnergy}
\end{align}

The spectroscopic constants in Equations \ref{eqn:vibrationalEnergy} and \ref{eqn:rotationalEnergy} are found in literature\cite{1995-zachwieja} and are provided here in Table \ref{tab:spectroscopicConstants}.

\begin{table}
  \caption[Spectroscopic constants for the CH \(X^2\Pi\) state]{Spectroscopic constants for the CH \(X^2\Pi\) state are presented.}
  \begin{center}
    \begin{tabular}{lr}
      Constant & Value \tabularnewline
      & cm\(^{-1}\) \tabularnewline
      \hline\hline
      & \tabularnewline
      \(\omega_e\) & 2860.7508 \tabularnewline
      \(\omega_ex_e\) & 64.4387 \tabularnewline
      \(\omega_ey_e\) & 0.36345 \tabularnewline
      \(\omega_ez_e\) & \(-1.5378 \times 10^{-2}\) \tabularnewline
      \(B_e\) & 14.459883 \tabularnewline
      \(\alpha_e\) & 0.536541 \tabularnewline
      \(D_e\) & \(1.47436 \times 10^{-3}\) \tabularnewline
      \(\beta_e\) & \(-2.530 \times 10^{-5}\) \tabularnewline
      \hline
    \end{tabular}
  \end{center}
  \label{tab:spectroscopicConstants}
\end{table}



The rovibrational partition function, \(Q_{rv}\) is a summation over all available vibrational and rotational levels in the particular electronic state.
For the ground state of the CH molecule, there are five available vibrational quantum numbers, \(v = 0\) to \(v = 4\).
The CH system falls under Hund's Case b and hence, the appropriate rotational quantum number to use is \(N\).
Each vibrational level has twenty-two possible values for \(N\) from \(N = 1\) to \(N = 22\).
For each rotational quantum number \(N\), there are two possible values of \(J\) given by \(N \pm \frac{1}{2}\).

