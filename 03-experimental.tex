\chapter{Experimental Methods and Considerations}
%   GENERALLY - YOU ARE QUITE BRIEF IN YOUR DESCRIPTIONS - 
% REMEMBER, YOU ARE SUPPOSED TO GIVE ENOUGH DETAILS FOR SOMEONE TO RECREATE THE EXPERIMENT AND TO UNDERSTAND HOW YOU DID IT


% It might be interesting to break up the chapter like this:
% 1. Experimental Facilities
%   a. LSB Test Rig A
%   b. LSB Test Rig B
%   c. Bunsen flame?
% 2. Diagnostic Techniques
%   a. LDV
%   b. CH* chemiluminescence
%   c. CH PLIF

The current chapter details the facilities and apparatus used to study the flame characteristics in a Low Swirl Burner.
The selection and implementation of diagnostic techniques used in this study are explained, as are data analysis methods used to process the acquired data.

\section{LSB configuration}

Two configurations of the Low Swirl Burner were tested for this study.
There are referred to in what follows as Configurations A and B.
Each configuration is built around a swirler with an outer diameter, \(d_s\) of 38 mm (1.5 in).
Other key dimensions of the swirlers tested for this work are presented in Table \ref{tab:swirlerdimensions}.

% TODO: Add dimensions for configuration B
\begin{table}

\caption[Swirler Dimensions]{The dimensions of the swirlers used and the respective perforated plates are presented. Each swirler is referred to by its vane angle (as in ``\(S_{37^\circ}\)'').}

\begin{center}
\begin{tabular}{lcc}
  Geometric parameter & \multicolumn{2}{c}{Swirler} \tabularnewline
  & \(S_{37^\circ}\) & \(S_{45^\circ}\) \tabularnewline
  \hline \hline
  \textbf{Swirler data} & & \tabularnewline
  Outer diameter, \(d_s\), mm & 38 & 38 \tabularnewline
  Diameter ratio, \(\frac{d_i}{d_s}\) & 0.66 & 0.66 \tabularnewline
  Vane angle, \(\alpha\) & 37\(^\circ\) & 45\(^\circ\) \tabularnewline
  Theoretical Swirl Number, \(S\) & 0.48 & 0.64 \tabularnewline
  & & \tabularnewline
  \textbf{Perforated plate data} & & \tabularnewline
  Open area, mm\(^2\) & 155.97 & 156.98 \tabularnewline
  Blockage, \% & 71.54 & 71.36 \tabularnewline
  Plate thickness, mm & 1.27 & 1.27 \tabularnewline
  Hole pattern & 1 - 8 - 16 & 1 - 8 - 16 \tabularnewline
  Hole location (dia), mm & 0 - 10.2 - 19.1 & 0 - 10.2 - 19.1 \tabularnewline
  Hole diameter, mm & 2.79 - 2.79 - 2.84 & 2.82 - 2.82 - 2.83 \tabularnewline
\end{tabular}
\end{center}

\label{tab:swirlerDimensions}

\end{table}



Initial testing aimed at velocity field mapping and flame imaging was conducted on Configuration A, while Configuration B was used for a later series of tests aimed at visualizing the flame structure.
The design of these two configurations is discussed in further detail in what follows.

\subsection{Configuration A}

In this configuration, the reactants reach the swirler through a converging nozzle that decreases linearly in diameter from from the inlet diameter of 102 mm (4 in) to the outer diameter of the swirler, 38 mm (1.5 in).
The swirler leads to a constant area nozzle, and is located one diameter upstream of an abrupt area change.
At the area change, the reactants expand from the 38 mm (1.5 in) diameter nozzle into a 115 mm (4.5 in) diameter combustion zone.
The expansion ratio is chosen so as to avoid confinement effects on the centerline flame flow field.\cite{1998-yegian}
%  SHOULDN'T YOU DESCRIBE THE FLOW ONLY PARTLY PASSING THRU THE SWIRLER, THE REST GOING THRU A CENTER PERF PLATE?

The main combustion zone begins at the dump plane and is enclosed by a GE 214 quartz tube.
The quartz tube is 300 mm (12 in) long and 115 mm (4.5 in) in diameter.
The thickness of the quartz tube is 2.5 mm (0.1 in).
Configuration A is illustrated in Figure FIXME.
%   INTRODUCE FIGURE EARLIER - E.G., AT START OF DESCRIPTION

\subsection{Configuration B}

%   YOU KEEP KEEP DESCRIBING THE WHOLE DEVICE AS THE SWIRLER, AS OPPOSED TO SWIRLER BEING THE OUTER ANNULUS THAT CAUSES THE SWIRL...I THINK THIS IS CONFUSING

In configuration B, the reactants reach the swirler through two separate streams.
The core stream passes through a layer of ball bearings before passing through a smoothly contoured nozzle with a high contraction ratio.
The annular stream passes directly through a smoothly contoured nozzle.
The contractions are designed to inhibit the formation of thick boundary layers.
The core stream passes through the central section of the swirler, while the annular stream picks up swirl by passing through the vanes of the swirler.

The swirler is located upstream of a constant area nozzle which is FIXME in length.
The reactants then expand into the combustion zone.

Unlike in Configuration A, there is no dump plane or quartz tube to provide confinement to the combustion zone.
The walls of the pressure vessel are insulated from the combustion zone by a co-flow of cold air.
Further, in this configuration, the annular flow is separately controlled from the central flow, which allows one to control the mass flow split directly.
Finally, this configuration allows for adjusting the level of turbulence present in the inlet flow by use of a turbulence generator located upstream in the plenum chamber.
%   ADD REFERENCE REGARDING TURBULENCE GENERATOR

The details of Configuration B are shown in Figure FIXME.
%   INTRODUCE FIGURE EARLIER 

\section{High Pressure Test Rig}
%   THE READER MIGHT BE CONFUSED WITH THIS SEPARATION METHOD - YOU COULD ALSO DESCRIBE EXPERIMENTAL SETUP A (SWIRLER AND RIG) AND THEN SETUP B

Each of the two configurations is housed in a separate high pressure testing rig with optical access to study the flame.
These rigs consist of an air and fuel supply system, a pressure vessel with adequate optical access and an exhaust system.
The details of each rig are discussed in the following sub-sections.

\subsection{Test Rig A}

Preliminary experiments involving velocity field mapping and flame imaging were conducted in Test Rig A, shown in Figure FIXME.
%   IS THE HOT AIR DRAWN FROM TANKS - OR IS AIR DRAWN FROM EXTERNAL TANKS AND HEATED IN AN INDIRECT, GAS-FIRED HEAT EXCHANGER.... 
Preheated air at about 500 K is drawn from external tanks and metered through an orifice flow meter.
The air enters the inlet nozzle of the LSB through a 1.8 m (6 ft) long, 102 mm (4 in) diameter straight pipe section.
%   NO DETAILS ON THE SPECIFIC METER, FLOW RATE RANGE, ETC.??
Fuel (natural gas) is metered using another orifice flow meter and injected at the head of the straight pipe section.
The straight pipe section allows for the flow to be fully developed, and fully premixed before the reactants enter the burner.

The combustor pressure and temperature are measured at the head of the inlet nozzle by a pressure transducer and a thermocouple respectively.
%   AGAIN - NO DETAILS ON THE TRANSDUCERS USED?
In addition, the upstream pressure and the pressure differential are measured at the air and fuel orifice flow meters.
For the preheated air stream, the upstream temperature is also measured.
% HOW MEASURED??
The measurements are used to calculate the four primary flow parameters (combustor pressure, preheat temperature, reference velocity and equivalence ratio) for the LSB in real time.
All measurements are monitored and recorded during the course of the experiment by a LabView VI.

The pressure vessel enclosing the combustor is designed to withstand pressures of up to 30 atm and is insulated from the combustor by a ceramic liner.
Cooling for the pressure vessel and the quartz tube is provided by a flow of cold air introduced at the head of the pressure vessel.
Optical access to the combustor is provided through four 25 mm (1 in) thick, 150 mm (6 in) \(\times\) 75 mm (3 in) quartz windows located \(90^\circ\) apart azimuthally.
The view ports allow the combustor to be imaged from the dump plane to an axial distance of 150 mm (6 in) downstream.

The exhaust from the combustor is cooled by circulating cold water through a water jacket enclosing the exhaust pipe section.
The length of the exhaust pipe section is about FIXME.
The exhaust pipe section terminates in an orifice plug to provide the back pressure to the combustion chamber.
Different diameter orifices are used for each reference velocity condition to be tested.
The exiting products are finally released to the building exhaust system.

\subsection{Test Rig B}

A schematic of Test Rig B is shown in Figure FIXME.
Both test rigs share the same upstream supply of preheated air, cold air and natural gas.
The preheated air splits into two separate streams a short distance after mixing with the natural gas fuel.
Orifice flow meters are used to meter the air and fuel flows prior to mixing.
%  DETAILS???
Further, each reactant stream is individually metered by separate orifice flow meters.
This builds redundancy in the system, offering a double-check of all readings and verifies that there are no leaks in the flow system.
Each orifice flow meter is equipped with a thermocouple, an upstream pressure transducer and a differential pressure transducer.
The cooling air for the co-flow is not metered.
All measurements are monitored and recorded by a LabView VI.

The pressure vessel is rated for pressures in excess of 30 atm and is insulated from the combustor by a flow of cold air.
The cold air enters the pressure vessel through two inlet ports and passes through a layer of steel ball bearings which renders the flow uniform spatially.
The pressure vessel has four viewports located \(90^\circ\) apart for optical access.
Each viewport is covered by a 25 mm (1 in) thick, 178 mm (7 in) \(\times\) 50 mm (2 in) quartz window.
The inlet of the LSB is located approximately halfway between the top and bottom edges of the window.
% DO YOU MEAN THE LSB EXIT
Similar to Test Rig A, the exhaust section is cooled by circulating cold water through an enclosing water jacket.
An adjustable gate valve on the exhaust line provides the back pressure necessary to pressurize the combustor.
The products are vented into the building exhaust system.
% ISN'T IN THE SAME EXHAUST SYSTEM THE OTHER RIG WENT IN TO?
\section{Diagnostics}

\subsection{Laser Doppler Velocimetry}

The velocity field of the LSB is mapped using a TSI 3-component LDV system.
Three wavelengths (514 nm, 488 nm and 476 nm) are separated from the output of a 5 W Argon ion laser by an FBL-3 multicolor beam generator.
The individual beams are split into two coherent beams which are then focused to intersect and produce interference fringes within an ellipsoidal measurement volume with dimensions of the order of 100 \(\mu\)m.
For this purpose, two transceiver probes are mounted \(90^\circ\) apart about the axis of the LSB.
One transceiver probe focuses the 514 nm and 488 nm beams in planes perpendicular to each other, while the second probe focuses the 476 nm beams orthogonal to the other two beams.
Particles in the flow field crossing the interference fringes scatter the laser light elastically and produce a sinusoidal signal whose frequency is proportional to the velocity of the particle.
The transceiver probes collect this scattered light and each wavelength is detected separately by a PDM-1000-3 three-channel photodetector module.
The output from the photodetector is processed by an FSA-3500-3 signal processor.
The resulting three components of the particle/flow velocity are recorded by the FlowSizer software.

Since the airflow is very sparsely populated by particles, the flow needs to be artificially seeded to facilitate LDV measurements in a reasonable amount of time.
The seeding particles to be used and their mean diameter are decided by the characteristics of the flow to be imaged.\cite{1997-melling}
Since the LSB flow field is a reacting one, the particles need to have high melting points.
Further, the particles need to be small enough to follow the flow closely and large enough or reflective enough to scatter light efficiently in the measurement volume.
Based on these requirements, commercially available alumina particles with a mean particle diameter of 5 \(\mu\)m were chosen for this study.
In order to uniformly seed the flow, a novel seeding generator was designed as described in Appendix \ref{app:seeder}.
The seeding particles were introduced slightly upstream of the 1.8 m (6 ft) long straight pipe section in Test Rig A.

LDV data was only acquired at atmospheric pressure conditions.
At high pressure conditions, the reacting LSB flow field produces sharp refractive index gradients that rapidly shift in the turbulent flow field.
This causes strong beam steering effects making it very difficult for the laser beams to reliably intersect within such a small measurement volume.
The long distance traveled by the beams in the test rig further exacerbated this problem, making LDV data nearly impossible to acquire at such conditions.

\subsection{CH* chemiluminescence}

The LSB flame is imaged using one of two 16-bit intensified CCD cameras --- PI Acton 1024\(\times\)256 or 512\(\times\)512 pixels --- with a 28 mm f/2.8 camera lens.
CH* chemiluminescence is filtered using a bandpass filter centered on 430 nm with a FWHM of 10 nm.
At each operating condition, 100 instantaneous images are acquired with an exposure of 1 ms.
An additional 100 instantaneous images are acquired with no flame and averaged to yield the background for correcting the flame images.

CH* chemiluminescence has several advantages over flame chemiluminescence from other radicals such as OH*, \ce{C2}*, etc.
First, the CH* emission occurs around 430 nm and is less affected by blackbody radiation from the walls of the combustor compared to longer wavelength detection, e.g., \ce{C2}*, which emits around 514 nm.
Second, the intensity of the chemiluminescence from CH* is known to scale well with heat release in the combustor\cite{2006-hardalupas}, unlike \ce{C2}*.
Third, the emitted light can be gathered with high quantum efficiency by the intensified CCD cameras used for this study.
The quantum efficiency of the 18 mm Gen III HB filmless intensifier used by the 512\(\times\)512 camera is about 45\% at 430 nm, compared to about 10\% at 310 nm, where OH* chemiluminescence peaks.

\subsubsection{Image Processing}

The flame chemiluminescence images acquired are background-corrected and averaged.
The resulting mean is the line-of-sight integrated, time-averaged image of the flame.
Strictly speaking, this is not the same as a real average obtained from a long exposure image.
The instantaneous images are obtained through a periodic sampling process and hence, are prone to statistical errors and aliasing.
% THERE IS NO ALIASING ISSUE IN THE AVERAGE - ONLY WHEN YOU TRY TO RECONSTRUCT THE TIME-DEPENDENCE, UNLESS THE FLOW IS PERIODIC AND THE DATA ACQUISITION RATE IS VERY CLOSE TO THE FREQUENCY OF THE FLOW PERIODCITY.
However, the behaviour of the flame can be assumed to be sufficiently random, and the mean obtained is adequately representative of the true average.
Figure FIXME shows a typical mean CH* chemiluminescence image prepared in this manner.

Even when background-corrected, the walls of the combustor are not at zero intensity in the average chemiluminescence image.
This is particularly noticeable near the dump plane where there is no flame present and yet the walls are clearly illuminated.
% YOU NEED TO BE MORE SPECIFIC IN YOUR DESCRIPTION BELOW - DO YOU MEAN the source of the illumination is mostly chemiluminescence from the flame scattering off the combustor walls...? AND HOW DO YOU KNOW THIS, WHY CAN'T IT BE THE HOT WALLS MOSTLY? 
The source of this illumination is mostly scattered chemiluminescence from the flame itself, and to a lesser degree, blackbody radiation from the heated walls.
The averaged chemiluminescence image allows us to measure the flame standoff distance by following the intensity profile along the centerline of the combustor.
The intensity profile rises sharply when passing the flame standoff location.
Thus, the flame standoff location can be ascertained by finding the inflection point in the intensity profile.

The profile of the average chemiluminescence intensity along the centerline of the sample case from Figure FIXME is shown in Figure FIXME, showing the flame standoff distance.
The distance from the dump plane, measured in number of pixels on the image and scaled by the appropriate magnification factor yields the flame standoff distance, \(X_f\).
The determination of the flame standoff location by this method provides a suitable and deterministic means to locating the leading edge of the flame front.
\nomenclature{\(X_f\)}{Flame standoff distance}

The average image can be processed further to yield more spatially resolved information about the flame brush.
Under the reasonable assumption that the average LSB flame is axially symmetric about the centerline of the combustor, a tomographic deconvolution technique called an Abel deconvolution\cite{1992-dasch} can be used to convert the line-of-sight integrated image to a radial map of chemiluminescence intensity.
In effect, this shows the shape and structure of the average flame brush.
The Abel deconvolution of the sample data from Figure FIXME is shown in Figure FIXME.

The Abel-deconvoluted image provides an relatively easy means to determining the angle of the flame brush.
A straight line joining two points located at the center of the flame brush intersects the axis of the combustor at this angle.
The angle of the flame is denoted by \(\theta_f\).

Using the Abel deconvolution to study the flame brush suffers from two main drawbacks.
First, the system of equations describing the Abel deconvolution is only valid as long as the entirety of the flame is visible.
This is only satisfied in the initial region of the LSB where the diameter of the flame brush is smaller than the height of the optical viewport.
At further downstream locations, the flame is not imaged in its entirety.
This causes the spurious bright regions near the top of the window in Figure FIXME.
The second limitation of the Abel deconvolution technique stems from the high incidence of errors along the centerline (where \(r \to 0\)).
Due to this, any study of the flame brush thickness at the flame stabilization point --- a metric of considerable importance --- is all but impossible using this tomographic technique.

\subsection{CH Planar Laser-Induced Fluorescence}
% HOW ABOUT SCHEMATIC OF LASER, DOUBLER, ETC.

The CH PLIF setup uses the frequency-doubled output of a Light Age PAL 101 alexandrite laser tuned to \(\lambda \approx 387.2\) nm to pump the R-bandhead of the CH \(B^2\Sigma^- \leftarrow X^2\Pi\) (0,0) system.
This populates the \(A^2\Delta\) state through fast electronic energy transfer from the \(B^2\Sigma^-\) state.
The resulting broadband fluorescence observed between \(\lambda\) = 420--440 nm is due to the \(A^2\Delta \rightarrow X^2\Pi\) (1,1), \(A^2\Delta \rightarrow X^2\Pi\) (0,0) and \(B^2\Sigma^- \rightarrow X^2\Pi\) (0,1) bands.
The CH PLIF signal is collected using an intensified PI Acton 512\(\times\)512 camera equipped with an 18 mm Gen III HB filmless intensifier with a quantum efficiency of about 45\% in the 420--440 nm range.
Elastic scattering from the laser beam is attenuated by a 3 mm thick GG 420 Schott Glass filter.

\subsubsection{Laser Wavelength Calibration}

% HOW ABOUT SCHEMATIC OF EXPERIMENTAL SETUP
The output of the PAL 101 alexandrite laser is controlled using a micrometer-coupled birefringent tuning mechanism.
The output wavelength of the laser varies linearly with the micrometer reading.
Initially, the manufacturer-supplied calibration for the micrometer was found to be inaccurate.
This required the recalibration of the laser in order to know the slope and offset of the calibration curve.

An Ocean Optics HR 2000 spectrometer was used to get the true wavelength of the laser output.
The spectrometer is pre-calibrated using 50 wavelengths in the 400--850 nm range from output of an Neon discharge lamp source.
The spectrometer is also intensity corrected over this range using a black body source.
The estimated error in the resolution of the device is about 0.1 nm (1 \AA).

The laser micrometer was traversed from 0.600 in to 0.626 in and back in steps of 0.001 in.
The calibration was performed using the fundamental wavelength of the laser.
Spectra were recorded at these conditions, integrated over 512 ms, and averaged over 10 acquisitions.
Each acquired spectrum was modeled as a Gaussian and the location of the center wavelength was noted.
The variation of the wavelength is verified to be linear against the micrometer setting and the correct calibration equation is obtained by doing a linear curve fit of the data.
The results are shown in Figure FIXME.

\subsubsection{Excitation scan}
% HOW ABOUT SCHEMATIC OF EXPERIMENTAL SETUP

An excitation scan was performed by tuning the output of the laser from \(\lambda\) = 387.077 nm to 387.260 nm.
This serves two purposes.
First, it locates the optimal wavelength to excite the CH radicals that results in the highest fluorescence yield.
Second, the variation of the signal intensity can be compared with simulated profiles from LIFBASE and our calculation of the laser linewidth can be validated.
% I DON'T THINK YOU EVER PRESENTED YOUR CALCULATION - NOT IN THIS CHAPTER AT LEAST

In order to do this, the CH PLIF imaging system, consisting of the alexandrite laser and the intensified PI Acton 512\(\times\)512 camera was used to image a premixed, stoichiometric, laminar methane-air flame.
The laminar flame is stabilized on a Bunsen burner with an inner diameter of 10.16 mm (0.4 in).
The alexandrite laser was operated at 10 Hz, with a power of 16 mJ/pulse.
% DESCRIBE WHAT OPTICS WERE USED
The second harmonic beam was converted into a sheet about 25 mm tall, with a thickness on the order of 100 \(\mu\)m passing through the center of the flame.
The edges of the sheet were blocked by razor blades.
% EXPLAIN WHY YOU DID THIS

The induced fluorescence is imaged perpendicularly by the intensified camera using an 85 mm f/1.8 Nikon AF Nikkor lens with an intensifier gate duration of over 300 ns.
The resolution of the imaging system is approximately 62 \(\mu\)m per pixel.
% IS THIS THE RESOLUTION OR THE PIXEL MAGNIFICATION
One hundred instantaneous images were acquired for each case.

Figure FIXME shows a sample CH PLIF image from this dataset.
The images are background-corrected and the signal statistics are measured at the midpoint of the flame.
% YOU ARE JUMPING TO FAST...YOU HAVEN'T TOLD THE READER WHAT SIGNAL STATISTICS YOU ARE TALKING ABOUT
The mean of the CH PLIF signal is compared with the LIFBASE-simulated profile of CH excitation LIF output.
The two signals are normalized by their individual maxima and plotted in Figure FIXME.
The profiles agree extremely well, validating our calibration of the laser output and our calculation of the laser linewidth.
The optimal excitation wavelength is found to be about 387.2 nm.
% SPEND SOME MORE TIME ABOVE - WHAT DO THE READERS SEE IN THE FIGURES, HOW GOOD IS THE COMPARISON - IS IT THE POSITION OF PEAKS, THE RELATIVE HEIGHTS, ETC. THAT MAKE SENSE....

\subsubsection{Linearity test}
% HOW ABOUT SCHEMATIC OF EXPERIMENTAL SETUP

% THIS NEXT PARAGRAPH DOES NOT FIT IN HERE - THE WRONG MOTIVATION AT THIS POINT. YOU CAN JUST SAY YOU ARE CHECKING LINEARITY OF THE FLUORESCENCE TO THE LASER INPUT
The next step is to estimate the variation of the LIF signal as a function of the operating conditions.
However, this function depends on which LIF regime we operate in.
Hence, it is imperative that we verify the regime of operation before attempting to model the CH PLIF signal.

% HAVE YOU INTRODUCED YOUR READER TO THIS IN THE BACKGROUND - OTHERWISE DO THEY KNOW WHAT YOU MEAN BY WEAK EXCITATION LIMIT...IF NOT, YOU NEED TO GIVE A BETTER BUILDUP
Under the assumption that the CH ground state population is not depleted by excitation or laser-induced chemical reactions, the LIF signal is linearly proportional to the laser intensity in the weak excitation limit.
However, as the laser intensity is increased further, the LIF output is observed to saturate and plateau.
This is called the strong excitation limit/saturation regime.
In the weak excitation limit, the signal is a function of CH concentration and the rate of collisional quenching of the excited CH radicals.
In the strong excitation limit, the signal depends only on the CH concentration is unaffected by the quenching of the excited CH species.

It is difficult to ensure that the CH system is saturated spatially, temporally and spectrally at the same time.
Further, operating with high laser intensities may bleach the energy levels being excited by inducing chemical reactions that destroy the excited CH radicals.
Hence, it is preferred to operate in the linear regime.

For this experiment, the laser beam is directed at a steady, laminar, methane-air, Bunsen flame operating at a slightly rich stoichiometry.
The 1 mm diameter beam is passed through an aperture, but no optics are used to refract the beam otherwise.
Varying the intensity of the laser beam by changing the flash lamp voltage or the Q-switch timing is not preferred as either would alter the pulse-width of the beam.
Instead, quartz disks and blocks are introduced into the beam to produce an intensity loss through reflection, scattering and absorption.

The flame is imaged with the PI Acton 512\(\times\)512 intensified camera equipped with a 50 mm, f/1.8 AF Nikkor lens and a 3 mm thick GG 420 filter.
% THIS IS NOT RESOLUTION BUT EQUIVALENT PIXEL SIZE?
The resolution of the set up was measured to be about 44 \(\mu\)m/pixel.
The laser power was varied from 10 mJ/pulse to 0.5 mJ/pulse in the manner described earlier.
The LIF signal image was recorded over 150 accumulations.
The corresponding laser scattering image was also recorded at each power for better estimating the background.
The flame chemiluminescence was also recorded for the same purpose.

The average signal per pixel is plotted in arbitrary units against the laser intensity in Figure FIXME.
The results indicate that the linearity of the LIF signal is valid for all laser intensities under 1 J/cm\(^2\).

% AGAIN, TOO SUCCINCT IN THE DESCRIPTION OF THE RESULTS