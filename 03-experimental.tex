\chapter{Experimental Methods and Considerations}
%   GENERALLY - YOU ARE QUITE BRIEF IN YOUR DESCRIPTIONS - 
% REMEMBER, YOU ARE SUPPOSED TO GIVE ENOUGH DETAILS FOR SOMEONE TO RECREATE THE EXPERIMENT AND TO UNDERSTAND HOW YOU DID IT

The current chapter describes the facilities and apparatus used to study the flame characteristics in a Low Swirl Burner.
The selection and implementation of diagnostic techniques used in this study are explained, as are data analysis methods used to process the acquired data.

\section{LSB configurations}

Two configurations of the Low Swirl Burner were tested for this study.
There are referred to in what follows as Configurations A and B.
Each configuration consists of the reactant flow inlet, the swirler device, the conduit to the combustion zone and the combustion zone itself.
All swirlers tested for this work have an outer diameter, \(d_s\) of 38 mm (1.5 in).
Other key dimensions of the swirlers tested are presented in Table \ref{tab:swirlerDimensions}.

Each configuration is housed in a high pressure testing facility.
The testing facility consists of an air and fuel supply system, a pressure vessel with adequate optical access and an exhaust system for the products.
Each testing facility is instrumented to measure temperatures and pressures which are then used to calculate various flow parameters of interest.

The design of the configurations tested, along with that of their respective test facilities are discussed in greater detail in this section.

% TODO: Add dimensions for configuration B
\begin{table}

\caption[Swirler Dimensions]{The dimensions of the swirlers used and the respective perforated plates are presented. Each swirler is referred to by its vane angle (as in ``\(S_{37^\circ}\)'').}

\begin{center}
\begin{tabular}{lcc}
  Geometric parameter & \multicolumn{2}{c}{Swirler} \tabularnewline
  & \(S_{37^\circ}\) & \(S_{45^\circ}\) \tabularnewline
  \hline \hline
  \textbf{Swirler data} & & \tabularnewline
  Outer diameter, \(d_s\), mm & 38 & 38 \tabularnewline
  Diameter ratio, \(\frac{d_i}{d_s}\) & 0.66 & 0.66 \tabularnewline
  Vane angle, \(\alpha\) & 37\(^\circ\) & 45\(^\circ\) \tabularnewline
  Theoretical Swirl Number, \(S\) & 0.48 & 0.64 \tabularnewline
  & & \tabularnewline
  \textbf{Perforated plate data} & & \tabularnewline
  Open area, mm\(^2\) & 155.97 & 156.98 \tabularnewline
  Blockage, \% & 71.54 & 71.36 \tabularnewline
  Plate thickness, mm & 1.27 & 1.27 \tabularnewline
  Hole pattern & 1 - 8 - 16 & 1 - 8 - 16 \tabularnewline
  Hole location (dia), mm & 0 - 10.2 - 19.1 & 0 - 10.2 - 19.1 \tabularnewline
  Hole diameter, mm & 2.79 - 2.79 - 2.84 & 2.82 - 2.82 - 2.83 \tabularnewline
\end{tabular}
\end{center}

\label{tab:swirlerDimensions}

\end{table}



\subsection{Configuration A}

Preliminary experiments involving velocity field mapping and flame imaging were performed using this configuration.
The schematic of the high pressure test facility housing this configuration is shown in Figure FIXME, while the configuration itself is shown in greater detail in Figure FIXME.

\subsubsection{Test Facility}

Pressurized air is supplied from external tanks and heated in an indirect, gas-fired heat exchanger to about 500 K.
The flowrate of the air is metered using a sub-critical orifice flow meter with a 38 mm (1.5 in) bore diameter Flow-Lin orifice plate capable of metering a maximum flow rate of 2.2 kg/s (1 lb/s).
The orifice flow meter is instrumented with an Omega PX725A-1KGI pressure transmitter calibrated to a reduced pressure range of 0--2.758 MPa (0--400 psi), a shielded K-type thermocouple and an Omega PX771A-025GI differential pressure transmitter, calibrated to a reduced differential pressure range of 0--68.948 kPa (0--10 psid).
The fuel (natural gas) is metered using a similar set up as the air line, with a sub-critical orifice flow meter.
The orifice plate is a Flow-Lin orifice plate with a bore diameter of 13.46 mm (0.53 in), capable of metering a maximum flow rate of 0.22 kg/s (0.1 lb/s).
The upstream pressure is measured using an Omega PX725A-1KGI pressure transmitter (same as the air line) and the differential pressure is measured using a PX771A-100WDC differential pressure transmitter with a pressure range of 0--2.489 kPa (100 in \ce{H2O}).
The temperature of the fuel is assumed to be the same as the room temperature (300 K).

The air enters the inlet nozzle of the LSB through a 1.8 m (6 ft) long, 102 mm (4 in) diameter straight pipe section.
The straight pipe section allows for the flow to be fully developed, and fully premixed before the reactants enter the burner.
The combustor pressure and temperature are measured at the head of the inlet nozzle.
The pressure is measured by an Omega PX181B-500G5V pressure transducer with a pressure range of 0--3.45 MPa (0--500 psi), while the temperature is measured using a K-type thermocouple.

The pressure and temperature measurements are used to calculate the four primary flow parameters (combustor pressure, preheat temperature, reference velocity and equivalence ratio) for the LSB in real time.
All measurements are monitored and recorded during the course of the experiment by a LabView VI.

The pressure vessel enclosing the combustor is designed to withstand pressures of up to 30 atm and is insulated from the combustor by a ceramic liner.
Cooling for the pressure vessel and the quartz tube is provided by a flow of cold air introduced at the head of the pressure vessel.
The cold air is drawn from the same external tanks as the main air line, but bypassing the heating system.
The cold air flow is not metered, but its upstream pressure is coupled to the main air line so as to ensure a stead flow of cold air into the pressure vessel at all operating conditions.
Optical access to the combustor is provided through four 25 mm (1 in) thick, 150 mm (6 in) \(\times\) 75 mm (3 in) quartz windows located \(90^\circ\) apart azimuthally.
The view ports allow the combustor to be imaged from the dump plane to an axial distance of 150 mm (6 in) downstream.

The exhaust from the combustor is cooled by circulating cold water through a water jacket enclosing each section of the exhaust pipe.
The length of the exhaust pipe sections is about 1.8 m (6 ft).
The exhaust pipe section terminates in an orifice plug to provide the back pressure to the combustion chamber.
Different diameter orifices are used for each reference velocity condition to be tested.
The exiting products are finally released to the building exhaust system.

\subsubsection{Low Swirl Burner}

The detail of the LSB configuration is shown in Figure FIXME.
The premixed, preheated reactants reach the swirler through a converging nozzle that decreases linearly in diameter from from the inlet diameter of 102 mm (4 in) to the outer diameter of the swirler, 38 mm (1.5 in).
At the swirler, the flow splits into two streams --- one passing through the central section and another picking up swirl by flowing over the vanes in the annular region.
The relative flow split between the two streams is controlled by inducing blockage into the central flow by means of a perforated plate.
The swirler leads to a constant area nozzle, and is located one diameter upstream of an abrupt area change.
At the area change, the reactants expand from the 38 mm (1.5 in) diameter nozzle into a 115 mm (4.5 in) diameter combustion zone.
This expansion ratio is chosen so as to avoid confinement effects on the centerline flame flow field.\cite{1998-yegian}

The main combustion zone begins at the dump plane and is enclosed by a GE 214 quartz tube.
The quartz tube is 300 mm (12 in) long and 115 mm (4.5 in) in diameter.
The thickness of the quartz tube is 2.5 mm (0.1 in).

\subsection{Configuration B}

This configuration was used to image the flame structure of the LSB flame using CH PLIF.
A schematic of the flow system of the test facility is shown in Figure FIXME, while the LSB combustor itself is shown in greater detail in Figure FIXME.

\subsubsection{Test Facility}

% TODO: Fill in details of the instrumentation in the following block:
%% START
This test facility shares the upstream supply of preheated air, cold air and fuel (natural gas) with the one used in Configuration A.
The flow rate of the preheated air stream is measured using the same orifice flow meter system used in Configuration A --- albeit with a smaller 27.59 mm (1.0863 in) diameter bore Flow-Lin orifice plate.
The fuel flow rate is measured using a critical orifice with a diameter of 0.8128 mm (0.032 in), instrumented with a 0--FIXME pressure transmitter, a K-type thermocouple and a FIXME differential pressure transmitter with a range of 0--FIXME.
A short distance after mixing with the fuel (natural gas), the preheated reactants split into two separate streams.
Each reactant stream is metered using an identical set up of .

Further, each reactant stream is individually metered by separate orifice flow meters.
This builds redundancy in the system, offering a double-check of all readings and verifies that there are no leaks in the flow system.
Each orifice flow meter is equipped with a thermocouple, an upstream pressure transducer and a differential pressure transducer.

The cooling air for the co-flow is not metered.
%% END

All measurements are monitored and recorded by a LabView VI.

The pressure vessel is rated for pressures in excess of 30 atm and is insulated from the combustor by a flow of cold air.
The cold air enters the pressure vessel through two inlet ports and passes through a layer of steel ball bearings which renders the flow uniform spatially.
The pressure vessel has four viewports located \(90^\circ\) apart for optical access.
Each viewport is covered by a 25 mm (1 in) thick, 178 mm (7 in) \(\times\) 50 mm (2 in) quartz window.
The LSB exit is located approximately halfway between the top and bottom edges of the window, allowing about 88.9 mm (3.5 in) of the combustion zone to be imaged through the window.
Similar to Test Rig A, the exhaust section is cooled by circulating cold water through an enclosing water jacket.
An adjustable gate valve on the exhaust line provides the back pressure necessary to pressurize the combustor.
The products are vented into the same building exhaust system as Configuration A.

\subsubsection{Low Swirl Burner}

The design of this LSB configuration is presented in Figure FIXME.
As described earlier, the reactants reach the swirler device through two separate streams.
The core/central stream passes through plenum chamber which is filled with steel ball bearings before approaching the swirler through a smoothly contoured nozzle with a high contraction ratio.
The annular stream reaches the swirler directly through a separate contoured nozzle.
The contraction ratio is chosen to inhibit the formation of thick boundary layers in the reactant streams.
The core stream passes through the central section of the swirler, while the annular stream picks up swirl by passing through the vanes of the swirler.

The swirler device leads to a constant area nozzle which is FIXME in length.
Following this, the reactants expand into the combustion zone.

Unlike in Configuration A, there is no dump plane or quartz tube to provide confinement to the combustion zone.
The co-flow of cold air provides insulation to the walls of the pressure vessel.
Further, in this configuration, the annular flow is separately controlled from the central flow, which allows one to control the mass flow split directly.
Finally, the level of turbulence in the central flow can be adjusted by use of a turbulence generator\cite{2011-marshall} located upstream in the plenum chamber.

\section{Diagnostics}

\subsection{Laser Doppler Velocimetry}

The velocity field of the LSB is mapped using a TSI 3-component LDV system.
Three wavelengths (514 nm, 488 nm and 476 nm) are separated from the output of a 5 W Argon ion laser by an FBL-3 multicolor beam generator.
The individual beams are split into two coherent beams which are then focused to intersect and produce interference fringes within an ellipsoidal measurement volume with dimensions of the order of 100 \(\mu\)m.
For this purpose, two transceiver probes are mounted \(90^\circ\) apart about the axis of the LSB.
One transceiver probe focuses the 514 nm and 488 nm beams in planes perpendicular to each other, while the second probe focuses the 476 nm beams orthogonal to the other two beams.
Particles in the flow field crossing the interference fringes scatter the laser light elastically and produce a sinusoidal signal whose frequency is proportional to the velocity of the particle.
The transceiver probes collect this scattered light and each wavelength is detected separately by a PDM-1000-3 three-channel photodetector module.
The output from the photodetector is processed by an FSA-3500-3 signal processor.
The resulting three components of the particle/flow velocity are recorded by the FlowSizer software.

Since the airflow is very sparsely populated by particles, the flow needs to be artificially seeded to facilitate LDV measurements in a reasonable amount of time.
The seeding particles to be used and their mean diameter are decided by the characteristics of the flow to be imaged.\cite{1997-melling}
Since the LSB flow field is a reacting one, the particles need to have high melting points.
Further, the particles need to be small enough to follow the flow closely and large enough or reflective enough to scatter light efficiently in the measurement volume.
Based on these requirements, commercially available alumina particles with a mean particle diameter of 5 \(\mu\)m were chosen for this study.
In order to uniformly seed the flow, a novel seeding generator was designed as described in Appendix \ref{app:seeder}.
The seeding particles were introduced slightly upstream of the 1.8 m (6 ft) long straight pipe section in Test Rig A.

LDV data was only acquired at atmospheric pressure conditions.
At high pressure conditions, the reacting LSB flow field produces sharp refractive index gradients that rapidly shift in the turbulent flow field.
This causes strong beam steering effects making it very difficult for the laser beams to reliably intersect within such a small measurement volume.
The long distance traveled by the beams in the test rig further exacerbated this problem, making LDV data nearly impossible to acquire at such conditions.

\subsection{CH* chemiluminescence}
\label{sec:chemiluminescence}

The LSB flame is imaged using one of two 16-bit intensified CCD cameras --- PI Acton 1024\(\times\)256 or 512\(\times\)512 pixels --- with a 28 mm f/2.8 camera lens.
CH* chemiluminescence is filtered using a bandpass filter centered on 430 nm with a FWHM of 10 nm.
At each operating condition, 100 instantaneous images are acquired with an exposure of 1 ms.
An additional 100 instantaneous images are acquired with no flame and averaged to yield the background for correcting the flame images.

CH* chemiluminescence has several advantages over flame chemiluminescence from other radicals such as OH*, \ce{C2}*, etc.
First, the CH* emission occurs around 430 nm and is less affected by blackbody radiation from the walls of the combustor compared to longer wavelength detection, e.g., \ce{C2}*, which emits around 514 nm.
Second, the intensity of the chemiluminescence from CH* is known to scale well with heat release in the combustor\cite{2006-hardalupas}, unlike \ce{C2}*.
Third, the emitted light can be gathered with high quantum efficiency by the intensified CCD cameras used for this study.
The quantum efficiency of the 18 mm Gen III HB filmless intensifier used by the 512\(\times\)512 camera is about 45\% at 430 nm, compared to about 10\% at 310 nm, where OH* chemiluminescence peaks.

\subsubsection{Image Processing}

The flame chemiluminescence images acquired are background-corrected and averaged.
The resulting mean is the line-of-sight integrated, time-averaged image of the flame.
Strictly speaking, this is not the same as a real average obtained from a long exposure image as the instantaneous images are obtained through a periodic sampling process and hence, are prone to statistical errors.
However, the behaviour of the flame can be assumed to be sufficiently random that the mean obtained is adequately representative of the true average.
Figure FIXME shows a typical mean CH* chemiluminescence image prepared in this manner.

Even when background-corrected, the walls of the combustor are not at zero intensity in the average chemiluminescence image.
This is particularly noticeable near the dump plane where there is no flame present and yet the walls are clearly illuminated.
The source of this background illumination is mostly the chemiluminescence from the flame scattering off the combustor and pressure vessel walls.
The contribution from blackbody radiation from the heated walls is less significant in the narrow wavelength range imaged.
This is evident from images acquired immediately after a flame blowout which show the walls to be nearly dark.

The averaged chemiluminescence image allows us to measure the flame standoff distance by following the intensity profile along the centerline of the combustor.
The intensity profile rises sharply when passing the flame standoff location.
Thus, the flame standoff location can be ascertained by finding the inflection point in the intensity profile.

The profile of the average chemiluminescence intensity along the centerline of the sample case from Figure FIXME is shown in Figure FIXME, showing the flame standoff distance.
The distance from the dump plane, measured in number of pixels on the image and scaled by the appropriate magnification factor yields the flame standoff distance, \(X_f\).
The determination of the flame standoff location by this method provides a suitable and deterministic means to locating the leading edge of the flame front.
\nomenclature{\(X_f\)}{Flame standoff distance}

The average image can be processed further to yield more spatially resolved information about the flame brush.
Under the reasonable assumption that the average LSB flame is axially symmetric about the centerline of the combustor, a tomographic deconvolution technique called an Abel deconvolution\cite{1992-dasch} can be used to convert the line-of-sight integrated image to a radial map of chemiluminescence intensity.
In effect, this shows the shape and structure of the average flame brush.
The Abel deconvolution of the sample data from Figure FIXME is shown in Figure FIXME.

The Abel-deconvoluted image provides an relatively easy means to determining the angle of the flame brush.
A straight line joining two points located at the center of the flame brush intersects the axis of the combustor at this angle.
The angle of the flame is denoted by \(\theta_f\).

Using the Abel deconvolution to study the flame brush suffers from two main drawbacks.
First, the system of equations describing the Abel deconvolution is only valid as long as the entirety of the flame is visible.
This is only satisfied in the initial region of the LSB where the diameter of the flame brush is smaller than the height of the optical viewport.
At further downstream locations, the flame is not imaged in its entirety.
This causes the spurious bright regions near the top of the window in Figure FIXME.
The second limitation of the Abel deconvolution technique stems from the high incidence of errors along the centerline (where \(r \to 0\)).
Due to this, any study of the flame brush thickness at the flame stabilization point --- a metric of considerable importance --- is all but impossible using this tomographic technique.

\subsection{CH Planar Laser-Induced Fluorescence}

The CH PLIF setup uses the frequency-doubled output of a Light Age PAL 101 alexandrite laser tuned to \(\lambda \approx 387.2\) nm.
The design of the laser is shown schematically in Figure FIXME.
The active medium is a 150 mm (6 in) long, 5 mm (0.197 in) diameter alexandrite rod.
The rod is placed between two flashlamps within the resonator cavity formed by two spherical mirrors.
A birefringent tuning mechanism is placed within the resonator to allow the user to select the frequency of the output beam.
The tuning mechanism is coupled to a micrometer whose reading relates linearly to the output wavelength.
The tuning mechanism allows the fundamental wavelength to be varied between 720--780 nm, with peak gain at about 755 nm.
The resonator cavity also contains two Q-switches, which allow the laser to optionally operate in double-pulsed mode.
For this study, however, only one Q-switch was used and the laser was operated in single-pulsed mode only.

The diameter of the fundamental beam exiting the output coupler is reduced by a collimating telescope.
This is done in order to increase the efficiency of conversion of the frequency-doubling crystal.
The second harmonic portion of the beam is separated from the fundamental by a dichroic mirror and exits the laser.
The fundamental beam is terminated at a beam dump within the laser.

The alexandrite laser is capable of operating at frequencies of up to 15 Hz.
Laser power is controlled primarily by varying the voltage applied to the flash lamps.
When operating with a high flash lamp voltage, it is recommended that the frequency of pulsing be reduced to allow more time to dissipate the heat build up within the alexandrite rod.
All experiments conducted as part of this study operated the laser at 10.0 Hz.

The linewidth of the fundamental beam is determined by the manufacturer to be 150 GHz at \(\lambda\) = 775 nm.
Assuming the spectral profile of the laser to be a Gaussian, the linewidth of the frequency-doubled beam can be determined.
The Full Width at Half Max (FWHM) of a Gaussian curve scales linearly with the standard deviation of the curve.
When convoluted with itself, the new standard deviation is \(\sqrt{\sigma^2 + \sigma^2}\) or \(\sqrt{2}\) times that of the original curve.
Thus, the new linewidth is 150 \(\times\sqrt{2}\) = 212 GHz or 7.07 cm\(^{-1}\).
In wavelengths, this represents a spread of about 1.06 \AA.

% This paragraph may need to be moved/rewritten.
The near-UV beam exiting the laser is used to pump the R-bandhead of the CH \(B^2\Sigma^- \leftarrow X^2\Pi\) (0,0) system.
This populates the \(A^2\Delta\) state through fast electronic energy transfer from the \(B^2\Sigma^-\) state.
The resulting broadband fluorescence observed between \(\lambda\) = 420--440 nm is due to the \(A^2\Delta \rightarrow X^2\Pi\) (1,1), \(A^2\Delta \rightarrow X^2\Pi\) (0,0) and, to a lesser degree, from the \(B^2\Sigma^- \rightarrow X^2\Pi\) (0,1) bands.
The LIF signal is collected using an intensified PI Acton 512\(\times\)512 camera equipped with an 18 mm Gen III HB filmless intensifier with a quantum efficiency of about 45\% in the 420--440 nm range.
Elastic scattering from the laser beam is attenuated by a 3 mm thick GG 420 Schott Glass filter.

\subsubsection{Laminar Flame setup}
\label{sec:laminarflamesetup}

Preliminary experiments to evaluate the CH PLIF technique are performed on a laminar flame.
The choice of a laminar flame as the subject allows us to neglect effects of strain and turbulence on the flame.
Further, laminar flames are more readily simulated by reaction kinetics packages like Chemkin with high fidelity.

These experiments are conducted on an laminar, methane-air flame stabilized on an unpiloted Bunsen burner with an inner diameter of 10.16 mm (0.4 in).
The air flow rate was measured and regulated using a Dwyer rotameter with a range of 0--20 SCFH calibrated using a Ritter drum-type gas meter.
The natural gas flow rate was metered using a Matheson FM 1050 602 rotameter with a range from 0--1230 SCCM.
This flowmeter was calibrated using a Sensidyne Gilibrator 2 bubble flow meter system.

\subsubsection{Laser Wavelength Calibration}

As described earlier, the output wavelength of the PAL 101 alexandrite laser is controlled using a micrometer-coupled birefringent tuning mechanism.
The wavelength of the laser beam varies linearly with the micrometer reading.
Initially, the manufacturer-supplied calibration for the micrometer was found to be inaccurate.
This required an experiment to calibrate the laser output wavelength against the micrometer reading in order to determine the slope and offset of the calibration curve accurately.

A schematic of this experiment is shown in Figure FIXME.
The laser beam is glanced off a steel optical post and the scattered light is collected using a fiber-optic cable coupled to an Ocean Optics HR 2000 spectrometer.
The spectrometer is pre-calibrated using 50 wavelengths in the 400--850 nm range from the output of a Neon discharge lamp source.
The spectrometer is also intensity corrected over this range using a black body source.
The estimated error in the resolution of the device is about 0.1 nm (1 \AA).

The laser micrometer was traversed from 0.600 in to 0.626 in and back in steps of 0.001 in.
The calibration was performed using at the fundamental wavelength of the laser.
Each spectrum recorded is integrated over 512 ms and averaged over 10 such acquisitions.
The background-corrected peak of the spectrum is then modeled as a Gaussian and the location of the center of the Gaussian waveform is recorded.

The results from this experiment are shown in Figure FIXME.
The variation of the wavelength is verified to be linear against the micrometer reading.
Further, there is little difference between the measurements taken while increasing and decreasing the micrometer position.
This indicates that any effects of hysteresis in the micrometer position are minimal.
A linear fit is applied to the points on the graph and the correct calibration equation is thus obtained.

\subsubsection{Excitation scan}

An excitation scan is performed by tuning the output of the alexandrite laser from \(\lambda\) = 387.077 nm to 387.260 nm.
This serves two purposes.
First, it locates the optimal wavelength to excite the CH radicals that results in the highest fluorescence yield.
Second, the variation of the signal intensity can be compared with simulated profiles from LIFBASE or other spectroscopic calculations and our estimation of the laser linewidth can be validated.
Knowing the laser linewidth accurately is important for modeling the LIF signal.
This aspect will be discussed in further detail later in this thesis/chapter FIXME.

A schematic of the excitation scan experiment is shown in Figure FIXME.
The intensified PI Acton 512\(\times\)512 camera described in Section \ref{sec:chemiluminescence} is used to image a premixed, laminar methane-air flame operating at close to stoichiometric conditions.
The laminar flame is stabilized on the Bunsen burner described in Section \ref{sec:laminarflamesetup}.
The alexandrite laser is operated at a power of 16 mJ/pulse in the second harmonic.
The sheet forming optics consist of a +50 mm cylindrical lens and a +250 mm spherical lens placed 300 mm apart.
The optics form the beam into a collimated sheet about 25 mm (1 in) tall, focused to a thickness on the order of 100 \(\mu\)m at the flame location.
The sheet passes through the center of the flame and the edges of the sheet are blocked by razor blades to prevent reflections from the burner from saturating the camera.

The induced fluorescence in the flame sheet is imaged perpendicularly by the intensified camera using an 85 mm f/1.8 Nikon AF Nikkor lens.
This gives a magnification of approximately 62 \(\mu\)m/pixel.
The camera is triggered by the flash lamp sync signal from the laser system and the intensifier is gated over 300 ns, encompassing the 70 ns laser pulse.
The long gate width gives the intensifier enough time to prepare to receive the fluorescence, preventing signal loss due to irising.
The gate width is still short enough that minimal flame chemiluminescence or ambient lighting is recorded in the images.
100 instantaneous images are acquired for each excitation wavelength to acquire a good estimate of the mean fluorescence signal, \(\mu_{sig}\).

Figure FIXME shows a sample CH PLIF image from this dataset.
The images are background-corrected by subtracting the laser scattering (recorded without the flame).
The fluorescence signal is calculated from these images using three alternate approaches.

In Method I, two ``windows'' are identified that include the straight sections of the laminar flame.
The average fluorescence signal in each frame is calculated by taking the average of all the emitting pixels in the two windows.
A pixel is designated as an emitting pixel if its intensity exceeds the standard deviation of a typical background pixel by at least a factor of five.
The average of this value over all the frames is designated as the mean fluorescence signal, \(\mu_{sig}\).
In Method II, the intensity of the pixels is integrated over a straight line connecting the inner and outer edges of the flame.
The straight line is chosen along the beam so that the beam intensity does not vary along the integration path.
The integration is performed on the left and right arms of the flame, giving two readings per frame.
The mean of these values over all the frames is recorded as the mean fluorescence signal, \(\mu_{sig}\).
In Method III, the midpoints of the straight lines from Method II are located and the average of their intensities, over all the frames is recorded as the mean fluorescence signal, \(\mu_{sig}\).
The regions of interest for each of these methods is highlighted in Figure FIXME.

The result of this investigation is shown in Figure FIXME.
The calculated mean fluorescence signals from the three methods are plotted against a LIFBASE simulation of the absorption spectrum of the CH \(B-X\) transition.
The profiles are appropriately scaled to match the LIFBASE simulation at the maximum value and at the minimum value.
The LIFBASE simulation is performed for a thermalized system at 1800 K, at atmospheric pressure.
Further, the instrument linewidth is specified to be the same as our estimate of the laser linewidth (1.06 \AA).

The profiles of the calculated and scaled mean fluorescence signals are observed to agree extremely well with the LIFBASE simulation result.
The discrepancies between the three methods is minimal.

The results indicate that the optimal excitation wavelength, corresponding to the highest mean fluorescence signal, is about 387.2 nm.
For the rest of the experiments performed in this work, the laser is operated at this wavelength.
The results also help verify that the calibration of the micrometer is accurate and the wavelengths are precisely adjustable.
Finally, the results validate that our estimated laser linewith, 1.06 \AA, is accurate.
This value can now be used in subsequent calculations of the LIF signal levels.

\subsubsection{Linearity test}

% HOW ABOUT SCHEMATIC OF EXPERIMENTAL SETUP

% THIS NEXT PARAGRAPH DOES NOT FIT IN HERE - THE WRONG MOTIVATION AT THIS POINT. YOU CAN JUST SAY YOU ARE CHECKING LINEARITY OF THE FLUORESCENCE TO THE LASER INPUT
The next step is to estimate the variation of the LIF signal as a function of the operating conditions.
However, this function depends on which LIF regime we operate in.
Hence, it is imperative that we verify the regime of operation before attempting to model the CH PLIF signal.

% HAVE YOU INTRODUCED YOUR READER TO THIS IN THE BACKGROUND - OTHERWISE DO THEY KNOW WHAT YOU MEAN BY WEAK EXCITATION LIMIT...IF NOT, YOU NEED TO GIVE A BETTER BUILDUP
Under the assumption that the CH ground state population is not depleted by excitation or laser-induced chemical reactions, the LIF signal is linearly proportional to the laser intensity in the weak excitation limit.
However, as the laser intensity is increased further, the LIF output is observed to saturate and plateau.
This is called the strong excitation limit/saturation regime.
In the weak excitation limit, the signal is a function of CH concentration and the rate of collisional quenching of the excited CH radicals.
In the strong excitation limit, the signal depends only on the CH concentration is unaffected by the quenching of the excited CH species.

It is difficult to ensure that the CH system is saturated spatially, temporally and spectrally at the same time.
Further, operating with high laser intensities may bleach the energy levels being excited by inducing chemical reactions that destroy the excited CH radicals.
Hence, it is preferred to operate in the linear regime.

For this experiment, the laser beam is directed at a steady, laminar, methane-air, Bunsen flame operating at a slightly rich stoichiometry.
The 1 mm diameter beam is passed through an aperture, but no optics are used to refract the beam otherwise.
Varying the intensity of the laser beam by changing the flash lamp voltage or the Q-switch timing is not preferred as either would alter the pulse-width of the beam.
Instead, quartz disks and blocks are introduced into the beam to produce an intensity loss through reflection, scattering and absorption.

The flame is imaged with the PI Acton 512\(\times\)512 intensified camera equipped with a 50 mm, f/1.8 AF Nikkor lens and a 3 mm thick GG 420 filter.
% THIS IS NOT RESOLUTION BUT EQUIVALENT PIXEL SIZE?
The resolution of the set up was measured to be about 44 \(\mu\)m/pixel.
The laser power was varied from 10 mJ/pulse to 0.5 mJ/pulse in the manner described earlier.
The LIF signal image was recorded over 150 accumulations.
The corresponding laser scattering image was also recorded at each power for better estimating the background.
The flame chemiluminescence was also recorded for the same purpose.

The average signal per pixel is plotted in arbitrary units against the laser intensity in Figure FIXME.
The LIF signal is seen to increase with increasing laser intensity.
At laser intensities below 1 J/cm\(^2\), the datapoints can be fit to a linear curve with only marginal scatter.
However, above this intensity, the repeatability of the data was poor and the linear trend cannot be extended reliably.

Thus, it can be concluded that as long as the intensity of the laser sheet is kept below 1 J/cm\(^2\), the LIF signal scales linearly with the intensity.

% AGAIN, TOO SUCCINCT IN THE DESCRIPTION OF THE RESULTS
