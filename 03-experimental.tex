\chapter{Experimental Methods and Considerations}

The current chapter details the facilities and apparatus used to study the flame characteristics in a Low Swirl Burner.
The selection and implementation of diagnostic techniques used in this study are explained, as are data analysis methods used to process the acquired data.

\section{LSB configuration}

Two LSB configurations, A and B are tested for this study.
Each LSB configuration is built around a swirler with an outer diameter, \(d_s\) of 38 mm (1.5 in).
Other key dimensions of the swirlers tested for this work are presented in Table \ref{tab:swirlerdimensions}.

\begin{table}

\caption[Swirler Dimensions]{The dimensions of the swirlers used and the respective perforated plates are presented. Each swirler is referred to by its vane angle (as in ``\(S_{37^\circ}\)'').}

\begin{center}
\begin{tabular}{lcc}
  Geometric parameter & \multicolumn{2}{c}{Swirler} \tabularnewline
  & \(S_{37^\circ}\) & \(S_{45^\circ}\) \tabularnewline
  \hline \hline
  \textbf{Swirler data} & & \tabularnewline
  Outer diameter, \(d_s\), mm & 38 & 38 \tabularnewline
  Diameter ratio, \(\frac{d_i}{d_s}\) & 0.66 & 0.66 \tabularnewline
  Vane angle, \(\alpha\) & 37\(^\circ\) & 45\(^\circ\) \tabularnewline
  Theoretical Swirl Number, \(S\) & 0.48 & 0.64 \tabularnewline
  & & \tabularnewline
  \textbf{Perforated plate data} & & \tabularnewline
  Open area, mm\(^2\) & 155.97 & 156.98 \tabularnewline
  Blockage, \% & 71.54 & 71.36 \tabularnewline
  Plate thickness, mm & 1.27 & 1.27 \tabularnewline
  Hole pattern & 1 - 8 - 16 & 1 - 8 - 16 \tabularnewline
  Hole location (dia), mm & 0 - 10.2 - 19.1 & 0 - 10.2 - 19.1 \tabularnewline
  Hole diameter, mm & 2.79 - 2.79 - 2.84 & 2.82 - 2.82 - 2.83 \tabularnewline
\end{tabular}
\end{center}

\label{tab:swirlerdimensions}

\end{table}

Initial testing aimed at velocity field mapping and flame imaging is conducted on Configuration A, while Configuration B is used for a later series of tests aimed at visualizing the flame structure.
The design of these two configurations is discussed in further detail in what follows.

\subsection{Configuration A}

In this configuration, the reactants reach the swirler through a converging nozzle that decreases linearly in diameter from from the inlet diameter of 102 mm (4 in) to the outer diameter of the swirler, 38 mm (1.5 in).
The swirler leads to a constant area nozzle, and is located one diameter upstream of an abrupt area change.
At the area change, the reactants expand from the 38 mm (1.5 in) diameter nozzle into a 115 mm (4.5 in) diameter combustion zone.
The expansion ratio is chosen so as to avoid confinement effects on the centerline flame flow field.\cite{1998-yegian}

The main combustion zone begins at the dump plane and is enclosed by a GE 214 quartz tube.
The quartz tube is 300 mm (12 in) long and 115 mm (4.5 in) in diameter.
The thickness of the quartz tube is 2.5 mm (0.1 in).
Configuration A is illustrated in Figure FIXME.

\subsection{Configuration B}

In this configuration, the reactants approach the swirler through a smoothly contoured nozzle with a high contraction ratio designed to inhibit the formation of thick boundary layers.
The swirler again leads to a constant area nozzle which is FIXME diameters in length.
Following this, the reactants enter the combustion zone.

Unlike in Configuration A, there is no dump plane or quartz tube to provide confinement to the combustion zone.
Further, in this configuration, the annular flow is separately controlled from the central flow, which allows one to control the mass flow split directly, if needed.
Finally, this configuration allows for adjusting the level of turbulence present in the inlet flow by use of a turbulence generator located upstream.

The details of Configuration B are shown in Figure FIXME.

\section{High Pressure Test Rig}

Each of the two configurations is housed in a separate high pressure testing rig with optical access to study the flame.
These rigs consist of an air and fuel supply system, a pressure vessel with adequate optical access and an exhaust system.
The details of each rig are discussed in the following sub-sections.

\subsection{Test Rig A}

Preliminary experiments involving velocity field mapping and flame imaging are conducted in Test Rig A, shown in Figure FIXME.
Preheated air at about 500 K is drawn from external tanks and metered through an orifice flow meter.
The air enters the inlet nozzle of the LSB through a 1.8 m (6 ft) long, 102 mm (4 in) diameter straight pipe section.
Fuel (natural gas) is metered using another orifice flow meter and injected at the head of the straight pipe section.
The straight pipe section allows for the flow to be fully developed, and fully premixed before the reactants enter the burner.
The combustor pressure and temperature are measured at the head of the inlet nozzle by a pressure transducer and a thermocouple respectively.
In addition, the upstream pressure and the pressure differential are measured at the air and fuel orifice flow meters.
For the preheated air stream, the upstream temperature is also measured.
The measurements are used to calculate the four primary flow parameters (combustor pressure, preheat temperature, reference velocity and equivalence ratio) for the LSB in real time.
All measurements are monitored and recorded during the course of the experiment by a LabView VI.

The pressure vessel enclosing the combustor is designed to withstand pressures of up to 30 atm and is insulated from the combustor by a ceramic liner.
Cooling for the pressure vessel and the quartz tube is provided by a flow of cold air introduced at the head of the pressure vessel.
Optical access to the combustor is provided through four 150 mm (6 in) \(\times\) 75 mm (3 in) quartz windows located \(90^\circ\) apart azimuthally.
The view ports allow the combustor to be imaged from the dump plane to an axial distance of 150 mm (6 in) downstream.

The exhaust from the combustor is cooled by circulating cold water through a water jacket enclosing the exhaust pipe section.
The length of the exhaust pipe section is about FIXME.
The exhaust pipe section terminates in an orifice plug to provide the back pressure to the combustion chamber.
Different diameter orifices are used for each reference velocity condition to be tested.
The exiting products are finally released to the building exhaust system.

\subsection{Test Rig B}

FIXME

\section{Diagnostics}

\subsection{Laser Doppler Velocimetry}

The velocity field of the LSB is mapped using a TSI 3-component LDV system.
Three wavelengths (514 nm, 488 nm and 476 nm) are separated from the output of a 5 W Argon ion laser by an FBL-3 multicolor beam generator.
The individual beams are split into two coherent beams which are then focused to intersect and produce interference fringes within an ellipsoidal measurement volume with dimensions of the order of 100 \(\mu\)m.
For this purpose, two transceiver probes are mounted \(90^\circ\) apart about the axis of the LSB.
One transceiver probe focuses the 514 nm and 488 nm beams in planes perpendicular to each other, while the second probe focuses the 476 nm beams orthogonal to the other two beams.
Particles in the flow field crossing the interference fringes scatter the laser light elastically and produce a sinusoidal signal whose frequency is proportional to the velocity of the particle.
The transceiver probes collect this scattered light and each wavelength is detected separately by a PDM-1000-3 three-channel photodetector module.
The output from the photodetector is processed by an FSA-3500-3 signal processor.
The resulting three components of the particle/flow velocity are recorded by the FlowSizer software.

Since the airflow is very sparsely populated by particles, the flow needs to be artificially seeded to facilitate LDV measurements in a reasonable amount of time.
The seeding particles to be used and their mean diameter are decided by the characteristics of the flow to be imaged.
Since the LSB flow field is a reacting one, the particles need to have high melting points.
Further, the particles need to be small enough to follow the flow closely and large enough or reflective enough to scatter light efficiently in the measurement volume.
Based on these requirements, commercially available alumina particles with a mean particle diameter of 5 \(\mu\)m were chosen for this study.
In order to uniformly seed the flow, a novel seeding generator was designed as described in Appendix \ref{app:seeder}.
The seeding particles were introduced slightly upstream of the 1.8 m (6 ft) long straight pipe section in Test Rig A.

LDV data is only acquired at atmospheric pressure conditions.
At high pressure conditions, the reacting LSB flow field produces sharp refractive index gradients that rapidly shift in the turbulent flow field.
This causes strong beam steering effects making it very difficult for the laser beams to reliably intersect within such a small measurement volume.
The long distance traveled by the beams in the test rig further exacerbated this problem, making LDV data nearly impossible to acquire at such conditions.

\subsection{CH* chemiluminescence}

The LSB flame is imaged using one of two 16-bit intensified CCD cameras --- PI Acton 1024\(\times\)256 or 512\(\times\)512 pixels --- with a 28 mm f/2.8 camera lens.
CH* chemiluminescence is filtered using a bandpass filter centered on 430 nm with a FWHM of 10 nm.
At each operating condition, 100 instantaneous images are acquired with an exposure of 1 ms.
An additional 100 instantaneous images are acquired with no flame and averaged to yield the background for correcting the flame images.

CH* chemiluminescence has several advantages over flame chemiluminescence from other radicals such as OH*, \ce{C2}*, etc.
First, the CH* emission occurs around 430 nm and is not affected by blackbody radiation from the walls of the combustor.
\ce{C2}*, on the other hand, emits around 514 nm and is significantly affected by this issue.
Second, the intensity of the chemiluminescence from CH* is known to scale well with heat release in the combustor\cite{2006-hardalupas}, unlike \ce{C2}*.
Third, the emitted light can be gathered with high quantum efficiency by the intensified CCD cameras used for this study.
The quantum efficiency of the 18 mm Gen III filmless HBf intensifier used by the 512\(\times\)512 camera is about 45\% at 430 nm, compared to about 10\% at 310 nm, where OH* chemiluminescence peaks.

\subsubsection{Image Processing}

\subsection{CH Planar Laser Induced Fluorescence}

\subsubsection{Excitation scan}
