\chapter*{Summary}

The Low Swirl Burner (LSB) is a promising combustion technology that is of importance to the gas turbine industry due to its potential for low \ce{NO_x} emissions and robust flame stabilization.
In order to effectively deploy the LSB in existing and future gas turbines, it is important to study its performance at high pressure, high preheat conditions.
This thesis aims to quantify and explain the effect of parameters such as the combustor pressure and preheat temperature on the location and shape of the LSB flame.
Towards that goal, the work employs LDV for atmospheric pressure velocity field mapping and CH* chemiluminescence for high pressure flame imaging.

Supplementing the scope of chemiluminescence imaging, an improved implementation of CH PLIF has been investigated for studying the structure of hydrocarbon flames.
The CH PLIF technique is demonstrated on a laminar Bunsen flame and on the LSB flame at atmospheric conditions.
A four-level model of the fluorescing CH system is developed to predict the signal intensity in any hydrocarbon flame, over a wide range of pressures, temperatures and equivalence ratios.
The model requires profiles of species concentrations and temperature across a flame to calculate the signal.
This information is obtained from Chemkin simulations using the San Diego mechanism to solve for conditions across the flame surface.
The results from imaging the atmospheric pressure laminar flame are used to validate the behavior of the signal intensity as predicted by the model.
The model is further extended to explore the feasibility of using CH PLIF in several reacting mixtures, including ethane-air, propane-air and syngas-methane-air mixtures.

The results from the LSB flame investigation demonstrate that although the combustor provides reasonably robust flame stabilization at moderately high pressures, the stabilization weakens at high pressure (above 12 atm).
Another interesting result from the experiments is that the shape of the flame is more sensitive to the equivalence ratio at these conditions.
% YOU SHOULD SAY WHY

The CH LIF model predictions indicate that the flame signal levels are diminished by collisional quenching at high pressure conditions, limiting the use of the technique to increasingly narrow equivalence ratio ranges at these conditions.
This limitation is alleviated increasing the preheat of the reactant mixture.
The signal levels from methane-air mixtures are found to be comparable to ethane-air and propane-air mixtures.
High hydrogen-content syngases can be imaged with CH PLIF at reasonable signal levels if they are doped with a small quantity of methane.


