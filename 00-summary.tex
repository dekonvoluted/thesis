\chapter*{Summary}

The Low Swirl Burner (LSB) is a promising combustion technology that is of importance to the gas turbine industry due to its potential for low \ce{NO_x} emissions and robust flame stabilization.
In order to effectively deploy the LSB in existing and future gas turbines, it is important to study its performance at high pressure, high preheat conditions.
This thesis aims to quantify and explain the effect of parameters such as the combustor pressure and preheat temperature on the location and shape of the lifted LSB flame.
Towards that goal, the work employs LDV for atmospheric pressure velocity field mapping and CH* chemiluminescence for high pressure flame imaging.

Supplementing the scope of chemiluminescence imaging, an improved implementation of CH PLIF has been investigated for studying the structure of hydrocarbon flames.
The CH PLIF technique is demonstrated on a laminar Bunsen flame and on the LSB flame at atmospheric conditions.
A four-level model of the fluorescing CH system is developed to predict the signal intensity in any hydrocarbon flame, over a wide range of pressures, temperatures and equivalence ratios.
The model requires profiles of species concentrations and temperature across a flame to calculate the signal.
This information is obtained from Chemkin simulations using the San Diego mechanism to solve for conditions across the flame surface.
The results from imaging the atmospheric pressure laminar flame are used to validate the behavior of the signal intensity as predicted by the model.
The model is further extended to explore the feasibility of using CH PLIF in several reacting mixtures, including ethane-air, propane-air and syngas-alkane-air mixtures.

The results from the LSB flame investigation reveal that combustor provides reasonably robust flame stabilization at low and moderate values of combustor pressure and reference velocities.
However, at very high velocities and pressures, the balance between the reactant velocity and the turbulent flame speed shifts in favor of the former resulting in the flame moving downstream.
The extent of this movement is small, but indicates a tendency towards blow off at higher pressures and velocities that may be encountered in real world gas turbine applications.
Another interesting observation from the experiments points to the increased tendency of fuel-rich flames to behave like attached flames at high pressure.
This is due to the enhanced strength of the outer/toroidal recirculation zone which enhances feedback of heat and reactants to the base of the flame at these conditions.
These results raise interesting questions about turbulent combustion at high pressure as well as provide usable data to gas turbine combustor designers by highlighting potential problems.

The CH LIF model calculations show that the fluorescence signal is highly weakened at high pressure due to the decreased number density of CH molecules and the increased collisional quenching rate.
This restricts the use of this technique to increasingly narrow equivalence ratio ranges at high pressures.
The limitation is somewhat alleviated by increasing the preheat temperature of the reactant mixture which enhances the CH concentration in the flame.
The signal levels from methane-air flames are found to be comparable other alkane air flames.
When applied to syngas-alkane-air mixtures, the signal levels from high hydrogen-content syngas mixtures doped with methane are found to be high enough to make CH PLIF a feasible diagnostic to study such flames.
Finally, the model predicts that signal levels are unlikely to be significantly affected by the presence of strain in the flow field, as long as the flames are not close to extinction.

