\chapter{CH PLIF Signal Modeling and Validation}

% This chapter covers the modeling of the CH PLIF signal and contains the following sections:

% 0. Preliminary experiments
%   0.1 Excitation scan
%   0.2 Linearity test
% 1. Signal modeling
%     1.1 Constants, assumptions, simplifications
%     1.2 Comparison of models (GRI, USCv2, Syngas, C1-C3, San Diego, etc.) / CH₄+air results
% 2. Results
%   Unstrained, laminar flames
%     2.1 Methane + air flames
%         2.1.1 subsection comparison with experiments
%     2.2 C1-C3 alkanes flames
%     2.3 Syngas mixtures
%     2.4 Syngas + C1-C3 alkanes
%   Strained laminar flames
%     2.5 Strain effects

% Questions to be answered (from proposal):

% 1. Detailing the development of a CH PLIF system (covered in Chapter 2,3 and here)
% 2. Demonstration on an atmospheric pressure laminar flame
% 3. Validate with quenching models
% 4. Model signal for strained flames
% 5. Model signal for fuel mixtures

% TODO:Write a short introduction to the chapter to start it off

\section{CH PLIF Development/Preliminary Experiments}

The CH PLIF imaging system was evaluated for use in imaging hydrocarbon flames by performing two preliminary experiments.
First, an excitation scan was performed to locate the optimal wavelength to excite the CH radicals in a typical hydrocarbon flame.
The variation in this optimal wavelength with temperature and pressure was calculated and found to be within acceptable bounds.
Second, a test of the linearity of the LIF signal with respect to the incident laser intensity was performed.
The setup and results of these experiments are described in the following subsections.

\subsection{Excitation scan}
\label{sec:excitationscan}

An excitation scan is performed by tuning the output of the alexandrite laser from \(\lambda\) = 387.077 nm to 387.260 nm.
This serves two purposes.
First, it locates the optimal wavelength to excite the CH radicals that results in the highest fluorescence yield.
Second, the variation of the signal intensity can be compared with simulated profiles from LIFBASE or other spectroscopic calculations and our estimation of the laser linewidth can be validated.
Knowing the laser linewidth accurately is important for modeling the LIF signal.
This aspect will be discussed in further detail later in this thesis/chapter FIXME.

A schematic of the excitation scan experiment is shown in Figure FIXME.
The intensified PI Acton 512\(\times\)512 camera described in Section \ref{sec:chemiluminescence} is used to image a premixed, laminar methane-air flame operating at close to stoichiometric conditions.
The laminar flame is stabilized on the Bunsen burner described in Section \ref{sec:laminarflamesetup}.
The alexandrite laser is operated at a power of 16 mJ/pulse in the second harmonic.
The sheet forming optics consist of a +50 mm cylindrical lens and a +250 mm spherical lens placed 300 mm apart.
The optics form the beam into a collimated sheet about 25 mm (1 in) tall, focused to a thickness on the order of 100 \(\mu\)m at the flame location.
The sheet passes through the center of the flame and the edges of the sheet are blocked by razor blades to prevent reflections from the burner from saturating the camera.

The induced fluorescence in the flame sheet is imaged perpendicularly by the intensified camera using an 85 mm f/1.8 Nikon AF Nikkor lens.
This gives a magnification of approximately 62 \(\mu\)m/pixel.
The camera is triggered by the flash lamp sync signal from the laser system and the intensifier is gated over 300 ns, encompassing the 70 ns laser pulse.
The long gate width gives the intensifier enough time to prepare to receive the fluorescence, preventing signal loss due to irising.
The gate width is still short enough that minimal flame chemiluminescence or ambient lighting is recorded in the images.
100 instantaneous images are acquired for each excitation wavelength to acquire a good estimate of the mean fluorescence signal, \(\mu_{sig}\).

Figure FIXME shows a sample CH PLIF image from this dataset.
The images are background-corrected by subtracting the laser scattering (recorded without the flame).
The fluorescence signal is calculated from these images using three alternate approaches.

In Method I, two ``windows'' are identified that include the straight sections of the laminar flame.
The average fluorescence signal in each frame is calculated by taking the average of all the emitting pixels in the two windows.
A pixel is designated as an emitting pixel if its intensity exceeds the standard deviation of a typical background pixel by at least a factor of five.
The average of this value over all the frames is designated as the mean fluorescence signal, \(\mu_{sig}\).
In Method II, the intensity of the pixels is integrated over a straight line connecting the inner and outer edges of the flame.
The straight line is chosen along the beam so that the beam intensity does not vary along the integration path.
The integration is performed on the left and right arms of the flame, giving two readings per frame.
The mean of these values over all the frames is recorded as the mean fluorescence signal, \(\mu_{sig}\).
In Method III, the midpoints of the straight lines from Method II are located and the average of their intensities, over all the frames is recorded as the mean fluorescence signal, \(\mu_{sig}\).
The regions of interest for each of these methods is highlighted in Figure FIXME.

The result of this investigation is shown in Figure FIXME.
The calculated mean fluorescence signals from the three methods are plotted against a LIFBASE simulation of the absorption spectrum of the CH \(B-X\) transition.
The profiles are appropriately scaled to match the LIFBASE simulation at the maximum value and at the minimum value.
The LIFBASE simulation is performed for a thermalized system at 1800 K, at atmospheric pressure.
Further, the instrument linewidth is specified to be the same as our estimate of the laser linewidth (1.06 \AA).

The profiles of the calculated and scaled mean fluorescence signals are observed to agree extremely well with the LIFBASE simulation result.
The discrepancies between the three methods is minimal.

The results indicate that the optimal excitation wavelength, corresponding to the highest mean fluorescence signal, is about 387.2 nm.
For the rest of the experiments performed in this work, the laser is operated at this wavelength.
The results also help verify that the calibration of the micrometer is accurate and the wavelengths are precisely adjustable.
Finally, the results validate that our estimated laser linewith, 1.06 \AA, is accurate.
This value can now be used in subsequent calculations of the LIF signal levels.

\subsection{Linearity test}

% HOW ABOUT SCHEMATIC OF EXPERIMENTAL SETUP

% THIS NEXT PARAGRAPH DOES NOT FIT IN HERE - THE WRONG MOTIVATION AT THIS POINT. YOU CAN JUST SAY YOU ARE CHECKING LINEARITY OF THE FLUORESCENCE TO THE LASER INPUT

As explained in Chapter FIXME, the variation of the fluorescence signal with the excitation laser intensity exhibits a saturation curve.
In the weak excitation limit, the variation is linear and scales with the energy input.
For reasons explained in Chapter FIXME, it is preferred to operate in this linear regime.

An experiment is performed to verify the linearity of the system response at the intensities at which the flames are imaged for this work.
The schematic of the setup is shown in Figure FIXME.
The laser is tuned to the optimal wavelength as determined in Section \ref{sec:excitationscan}, and operated at 10 Hz.
The frequency-doubled beam is directed at a steady, laminar, methane-air Bunsen flame operating at a slightly rich stoichiometry.
The edges of the beam are clipped by an aperture to produce a sharp edge and to avoid unnecessary reflections from the burner.
No optics are used to refract the beam in any way.

The flame is imaged by the PI Acton 512\(\times\)512 intensified camera equipped with a 50 mm, f/1.8 AF Nikkor lens.
Elastic scattering is attenuated by a 3 mm thick GG 420 Schott glass filter.
The magnification achieved by this set up is about 44 \(\mu\)m/pixel.
The LIF signal from the flame is recorded in 300 ns gates and accumulated 150 times before being read out.
For each case, a corresponding laser scattering image is also recorded for estimating the background.
The flame chemiluminescence and ambient background are also recorded for the same purpose.

For this experiment, varying the intensity of the laser beam by changing the flash lamp voltage or even the Q-switch timing is not preferred as either would alter the pulse width of the beam.
Instead, quartz disks and blocks of varying thickness are introduced into the beam to produce an intensity loss, while preserving all other characteristics of the beam.
The quartz elements decrease the intensity of the laser beam through reflection, scattering and absorption.
The stray reflections and scattering from the quartz elements are contained by enclosing the elements in a box and preventing these from being recorded by the camera.
In this manner, the laser power is varied from 10 mJ/pulse to 0.5 mJ/pulse and back.

The acquired images are background-corrected and the intensity is conditionally averaged over pixels with a non-zero intensity in the region where the fluorescence occurs.
The average fluorescence intensity values thus obtained are plotted against the corresponding laser intensity and shown in Figure FIXME.
A sample image highlighting the region of interest is also shown alongside.

The LIF signal is observed to increase monotonically with increasing laser intensity.
At the lower intensities, the variation is very nearly linear, with marginal scatter and only one significant outlier.
At intensities above 1 J/cm\(^2\) however, there is significant scatter in the data and the linear trend obtained from the low intensity cases cannot be reliably extended over this region.

The results indicate that as long as the intensity of the laser sheet is kept below 1 J/cm\(^2\), the assumption of operating in the linear regime is valid.

\section{Fluorescence Signal Modeling}

In Chapter FIXME 2, we derived the expression for the LIF signal as a function of the thermodynamic conditions and the local composition.
The expression requires the pre-knowledge(?) of several spectroscopic constants of the CH system.



As described in Chapter FIXME 2, the excitation scheme used in this study produces fluorescence through a three-step process.
First, the CH radicals in the ground state \(X^2\Pi, v=0\) are excited by the incident radiation to the second electronically excited state \(B^2\Sigma^-, v=0\).
This excitation occurs near the R-bandhead and targets the ground state CH radicals present in the rotational energy levels, \(N=5\) through 9.
The upper electronic state \(B^2\Sigma^-, v=0\) is nearly degenerate with the \(A^2\Delta, v=1\) energy level.
This leads to the population of the \(A^2\Delta, v=0,1\) energy levels due to collisional energy transfer.
The resulting fluorescence collected is primarily the result of three spontaneous transitions --- \(A\rightarrow X(1,1)\), \(A\rightarrow X(0,0)\) and \(B\rightarrow X(0,1)\).
These transitions are shown in Figure FIXME.

The primary goal of this exercise of modeling the CH fluorescence signal intensity is to gage the feasibility of using CH PLIF to study various premixed flames, rather than to quantitatively calculate the amount of CH present in the flames.
As such, we are more interested in the order of magnitude of the PLIF signal, rather than the absolute value of it.

The intensity of the CH fluorescence signal may be written as a function of the amount of CH radicals present in the excited state and the probability of spontaneous emission from said state.
Symbolically, this may be written as shown in Equation \ref{eqn:signalIntensity}.

\begin{equation}
S=nVA
\label{eqn:signalIntensity}
\end{equation}

In Equation \ref{eqn:signalIntensity}, \(S\) is the total number of photons emitted per unit time, \(n\) is the number of excited CH radicals in a unit volume, \(V\) is the volume from which the signal is observed.
The Einstein coefficient for spontaneous emission, \(A\) represents the probability of spontaneous emission between the two involved energy states.
The predicted signal intensity represents the total number of photons emitted in all directions.
In reality, only a fraction of these emitted photons will be recorded by the collection system.
This fraction is a function of the experimental setup and depends on the collection angle, the efficiency of the optics and the detector used to record the signal.
This fraction is left out because our objective is only to predict the relative variation in the signal between various premixed flames.

This formulation of the signal intensity implicitly makes the following assumptions.
\begin{enumerate}
\item The fluorescence emission is predicted at steady state.
\item The collection volume is optically thin and an emitted photon is not reabsorbed within the flame itself.
This is a reasonable assumption to make, since the flame thickness and the thickness of the laser sheet are both typically quite small.
\end{enumerate}

As described earlier, an accurate model of the CH system should involve five energy levels --- \(X(0)\), \(B(0)\), \(A(1)\), \(A(0)\), and \(X(1)\)\footnote{In this notation, the letter represents the electronic energy level and the number in the parentheses represents the vibrational quantum number of the energy level}.
Such a model would also have to account for collisional transfers between each of these levels, in addition to spontaneous and stimulated transitions.
The mathematical solution quickly becomes complicated and tedious.
Further, it would involve several rate coefficients that have not been measured in experiments done so far.

\section{Results}

Comparison of CH concentration predicted by GRI Mech and San Diego mechanisms for methane.

