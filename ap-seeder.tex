\chapter{Seeder Design}
\label{app:seeder}

A new seeder was designed for use in high pressure implementations of diagnostic techniques like Laser Doppler Velocimetry (LDV), Particle Image Velocimetry (PIV), etc.

The previous design, as shown in Figure FIXME, was a fluidized bed seeding generator.
Seeding particles in a cylindrical vessel are fluidized by an air-turbine vibrator.
Air is introduced into the vessel in the form of two opposing jets directed tangentially to produce a small amount of swirl in the flow field.
Particles are picked up by the air flow and the swirl aids in separating the heavy/coagulated clumps of seeding particles by centrifugal acceleration.

This design had several shortcomings.
First, it is observed that the seeding density of the seeded flow generally decreases over time, even if the seeding particles have not been depleted.
The seeding particles tend to coagulate over time, due to the buildup of moisture, static charge, etc.
In such cases, the vibrator can no longer effectively fluidize the particles.
Further, the tangential introduction of the air flow preferentially depletes particles near the walls of the container, leaving the center relatively undisturbed.
The cumulative effect of these phenomena diminishes the effectiveness of the seeder.

Second, the fluidized bed requires a minimum amount of seeding particles to function effectively.
This requires the seeder to be refilled even before all the seeding particles are consumed.

Third, when designed for high pressure applications, the seeder will become quite heavy due to flanges and other fittings.
Such a setup cannot be easily fluidized using a reasonable-sized air-turbine vibrator.

The new seeder design is shown in Figure FIXME, and resembles a funnel with a swirler located halfway up the stem.
A perforated base plate holds the swirler and the seeding particles in the conical section of the swirler.
Due to the steep angle of the sides of the conical section, the seeding particles continuously collapse into the central section.
This negates any need for vibrating the system.
Air is introduced from the bottom of the seeder and enters the vessel by passing through the swirler.
Since all the air enters this way, there is a considerable amount of swirl in the resulting flow field,
Heavy/coagulated seeding particles are flung outward, while lighter particles are carried with the air.
After a sufficient distance to allow for the cyclonic separation to be effective, the seeded air passes through another perforated plate which further limits the presence of large clumps of particles.
The exiting air is now spatially and temporally uniformly seeded.

