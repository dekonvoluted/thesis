\begin{figure}

\centering

\hfill
\begin{subfigure}{0.45\linewidth}
  \input{figures/leanFlameAverageImage}
  \caption{\(\phi\) = 0.53}
  \label{fig:leanFlameAverageImage}
\end{subfigure}
\hfill
\begin{subfigure}{0.45\linewidth}
  \input{figures/richFlameAverageImage}
  \caption{\(\phi\) = 0.58}
  \label{fig:richFlameAverageImage}
\end{subfigure}
\hfill

\hfill
\begin{subfigure}{0.45\linewidth}
  \input{figures/leanFlameAbelImage}
  \caption{Abel-deconvoluted image}
  \label{fig:leanFlameAbelImage}
\end{subfigure}
\hfill
\begin{subfigure}{0.45\linewidth}
  \input{figures/richFlameAbelImage}
  \caption{Abel-deconvoluted image}
  \label{fig:richFlameAbelImage}
\end{subfigure}
\hfill

\caption[Effect of equivalence ratio on the flame shape - I]{Figures \ref{fig:leanFlameAverageImage}--\ref{fig:richFlameAverageImage} show mean CH* chemiluminescence images of the LSB flame taken at high pressure for two equivalence ratios. Figures \ref{fig:leanFlameAbelImage}--\ref{fig:richFlameAbelImage} show the Abel-deconvolution of the average image highlighting the flame brush. The centerline of the combustor is along the lower edge of the Abel half-images.}

\label{fig:equivalenceRatioResults}

\end{figure}

