\begin{figure}

\centering

\begin{subfigure}{\linewidth}
  \centering
  \input{figures/pressureDistancePlot}
  \caption{Flame standoff distance as a function of the combustor pressure}
  \label{fig:pressureDistance}
\end{subfigure}

\begin{subfigure}{\linewidth}
  \centering
  \input{figures/pressureAnglePlot}
  \caption{Flame cone angle as a function of the combustor pressure}
  \label{fig:pressureAngle}
\end{subfigure}

\caption[Effect of combustor pressure on the flame location and shape]{The plots above show the effect of varying the combustor pressure on the flame location and flame cone angle. The \textcolor{blue}{blue} curves represent data points from the \(S_{37^\circ}\) swirler tests, while the \textcolor{red}{red} curves represent data from the \(S_{45^\circ}\) swirler tests.}

\label{fig:pressureResults}

\end{figure}

