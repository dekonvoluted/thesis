\begin{figure}

\centering

\begin{subfigure}{\linewidth}
  \centering
  \input{figures/pressureDistancePlot}
  \caption{Flame standoff distance as a function of the reference velocity}
  \label{fig:pressureDistance}
\end{subfigure}

\begin{subfigure}{\linewidth}
  \centering
  \input{figures/pressureAnglePlot}
  \caption{Flame cone angle as a function of the reference velocityFIXME}
  \label{fig:pressureAngle}
\end{subfigure}

\caption[Effect of Combustor Pressure on Flame Location and Shape]{The plots above show the effect of changing the reference velocity on the flame location and flame shape. The \textcolor{blue}{blue} curves are from tests conducted using the \(S_{37^\circ}\) swirler at 6 atm, while the \textcolor{red}{red} curves are from tests using the \(S_{45^\circ}\) swirler at 3 atm. The preheat for all these cases was about 500 K.}

\label{fig:pressureResults}

\end{figure}

