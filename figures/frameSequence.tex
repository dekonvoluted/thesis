\begin{figure}

\centering

\hfill
\begin{subfigure}{0.45\linewidth}
  \centering
  \input{figures/frame90}
  \caption{Frame \#90}
  \label{fig:frame90}
\end{subfigure}
\hfill
\begin{subfigure}{0.45\linewidth}
  \centering
  \input{figures/frame91}
  \caption{Frame \#91}
  \label{fig:frame91}
\end{subfigure}
\hfill

\hfill
\begin{subfigure}{0.45\linewidth}
  \centering
  \input{figures/frame92}
  \caption{Frame \#92}
  \label{fig:frame92}
\end{subfigure}
\hfill
\begin{subfigure}{0.45\linewidth}
  \centering
  \input{figures/frame93}
  \caption{Frame \#93}
  \label{fig:frame93}
\end{subfigure}
\hfill

\hfill
\begin{subfigure}{0.45\linewidth}
  \centering
  \input{figures/frame94}
  \caption{Frame \#94}
  \label{fig:frame94}
\end{subfigure}
\hfill
\begin{subfigure}{0.45\linewidth}
  \centering
  \input{figures/frame95}
  \caption{Frame \#95}
  \label{fig:frame95}
\end{subfigure}
\hfill

\caption[Effect of equivalence ratio on the flame shape - II]{The sequence of frames above is taken from the high equivalence ratio data set and shows the flame intermittently attaching to the lip of the inlet.}

\label{fig:frameSequence}

\end{figure}

