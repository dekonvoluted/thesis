\begin{figure}

\centering

\hfill
\begin{subfigure}{0.45\linewidth}
  \centering
  \input{figures/referenceVelocityLowVelPLIFImage}
  \caption{Instantaneous PLIF image (Low Velocity case)}
  \label{fig:referenceVelocityLowVelPLIFImage}
\end{subfigure}
\hfill
\begin{subfigure}{0.45\linewidth}
  \centering
  \input{figures/referenceVelocityHighVelPLIFImage}
  \caption{Instantaneous PLIF image (High velocity case)}
  \label{fig:referenceVelocityHighVelPLIFImage}
\end{subfigure}
\hfill

\hfill
\begin{subfigure}{0.45\linewidth}
  \centering
  \input{figures/referenceVelocityLowVelPLIFEdge}
  \caption{Edges}
  \label{fig:referenceVelocityLowVelPLIFEdge}
\end{subfigure}
\hfill
\begin{subfigure}{0.45\linewidth}
  \centering
  \input{figures/referenceVelocityHighVelPLIFEdge}
  \caption{Edges}
  \label{fig:referenceVelocityHighVelPLIFEdge}
\end{subfigure}
\hfill

\hfill
\begin{subfigure}{0.45\linewidth}
  \centering
  \input{figures/referenceVelocityLowVelPLIFHistogram}
  \caption{\(\mu\) = 364; \(\sigma\) = 167}
  \label{fig:referenceVelocityLowVelPLIFHistogram}
\end{subfigure}
\hfill
\begin{subfigure}{0.45\linewidth}
  \centering
  \input{figures/referenceVelocityHighVelPLIFHistogram}
  \caption{\(\mu\) = 463; \(\sigma\) = 187}
  \label{fig:referenceVelocityHighVelPLIFHistogram}
\end{subfigure}
\hfill

\caption[Effect of Reference Velocity on Flame Structure]{The sequence of images on the left and the right pertain to the low and high velocity cases respectively. Each instantaneous frame of PLIF data is processed to detect edges and the statistics of the edge pixels in the central quarter of the image are plotted as a histogram.}

\label{fig:referenceVelocityPLIFResults}

\end{figure}

