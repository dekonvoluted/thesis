\begin{figure}

\centering

\begin{tikzpicture}

% Ground states
\draw [thick] ( 3, 0 ) -- ++( 4, 0 );
\node at ( 7, 0 ) [right] {\(X^2\Pi, v = 0\)};
\draw [thick] ( 3, 1 ) -- ++( 4, 0 );
\node at ( 7, 1 )  [right] {\(X^2\Pi, v = 1\)};

% First electronic states
\draw [thick] ( 0, 4 ) -- ++( 3, 0 );
\node at ( 0, 4 )  [left] {\(A^2\Delta, v = 0\)};
\draw [thick] ( 0, 5 ) -- ++( 3, 0 );
\node at ( 0, 5 )  [left] {\(A^2\Delta, v = 1\)};

% Second electronic state
\draw [thick] ( 7, 5 ) -- ++( 3, 0 );
\node at ( 10, 5 )  [right] {\(B^2\Sigma^-, v = 0\)};

% Transitions
\draw [blue, ->] ( 6, 0 ) -- ++( 3, 5 );

\draw [red, ->] ( 6.5, 5 ) -- ++( -3, 0 );
\draw [red, ->] ( 6.5, 4.75 ) -- ++( -3, -0.75 );

\draw [green, ->] ( 8, 5 ) -- ++( -2.25, -4 );
\draw [green, ->] ( 1, 4 ) -- ++( 3, -4 );
\draw [green, ->] ( 1, 5 ) -- ++( 3, -4 );

\end{tikzpicture}

\caption[Relevant transitions in a CH molecule]{Some of the important transitions between energy levels in a CH molecule are shown. The excitation of the CH molecules (\textcolor{blue}{blue}) is followed by collisional energy transfer processes (\textcolor{red}{red}) which populate additional energy levels. Spontaneous emission from some of these energy levels (\textcolor{green}{green}) is collected.}

\label{fig:energyLevels}

\end{figure}

