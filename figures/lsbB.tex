\begin{figure}

\centering

\begin{tikzpicture}

% Left half (lower)
\draw[pattern=north west lines] ( -1.5, 0 ) -- ++( 0, 4.65 ) -- ++( -0.5, 0 ) -- ++( 0, 0.05 ) -- ++( 0.55, 0 ) -- ++( 0, -0.2 ) -- ++( 0.45, 0 ) -- ++( 0, 0.5 ) .. controls +( 0, 1 ) and +( 0, -1 ) .. ++( 0.75, 2 ) -- ++( 0.05, 0 ) .. controls +( 0, -1 ) and +( 0, 1 ) .. ++( -0.75, -2 ) -- ++( 0, -5 ) -- ++( 0.75, 0 ) -- ++( 0, -1 ) -- ++( -3.75, 0 ) -- ++( 0, 1 ) -- cycle;

% Left half (upper)
\draw[pattern=north west lines] ( -2, 4.85 ) -- ++( 0.5, 0 ) -- ++( 0, 0.15 ) .. controls +( 0, 1 ) and +( 0, -1 ) .. ++( 1, 2 ) -- ++( 0, 1.5 ) -- ++( 0.05, -0.1 ) -- ++( 0, -1.4 ) .. controls +( 0, -1 ) and +( 0, 1 ) .. ++( -1, -2 ) -- ++( 0, -0.2 ) -- ++( -0.55, 0 ) -- cycle;

% Right half (lower)
\draw[pattern=north west lines] ( 1.5, 0 ) -- ++( 0, 4.65 ) -- ++( 0.5, 0 ) -- ++( 0, 0.05 ) -- ++( -0.55, 0 ) -- ++( 0, -0.2 ) -- ++( -0.45, 0 ) -- ++( 0, 0.5 ) .. controls +( 0, 1 ) and +( 0, -1 ) .. ++( -0.75, 2 ) -- ++( -0.05, 0 ) .. controls +( 0, -1 ) and +( 0, 1 ) .. ++( 0.75, -2 ) -- ++( 0, -5 ) -- ++( -0.75, 0 ) -- ++( 0, -1 ) -- ++( 3.75, 0 ) -- ++( 0, 1 ) -- cycle;

% Right half (upper)
\draw[pattern=north west lines] ( 2, 4.85 ) -- ++( -0.5, 0 ) -- ++( 0, 0.15 ) .. controls +( 0, 1 ) and +( 0, -1 ) .. ++( -1, 2 ) -- ++( 0, 1.5 ) -- ++( -0.05, -0.1 ) -- ++( 0, -1.4 ) .. controls +( 0, -1 ) and +( 0, 1 ) .. ++( 1, -2 ) -- ++( 0, -0.2 ) -- ++( 0.55, 0 ) -- cycle;

\draw ( -0.5, 8.5 ) -- ++( 1, 0 );
\draw ( -0.5, 8.4 ) -- ++( 1, 0 );

% Swirler
\draw ( -0.25, 7.01 ) rectangle ++( 0.5, 0.45 );
\draw ( -0.45, 7.125 ) -- ++( 0.2, 0.25 ) -- ++( -0.2, 0 ) -- ++( 0.2, -0.25 ) -- cycle;
\draw ( 0.45, 7.125 ) -- ++( -0.2, 0.25 ) -- ++( 0.2, 0 ) -- ++( -0.2, -0.25 ) -- cycle;

% Turbulence generator
\draw[red,pattern=north east lines] ( -0.1, -2 ) -- ++( 0, 6.5 ) -- ++( -0.75, 0 ) -- ++( 0, 0.2 ) -- ++( 1.7, 0 ) -- ++( 0, -0.2 ) -- ++( -0.75, 0 ) -- ++( 0, -6.5 ) -- cycle;

% Ball bearings (inside)
\foreach \x in { -0.825, -0.625, -0.425, -0.225 }
  \foreach \y in { 0.125, 0.325, ..., 2.125 }
    \draw[fill=black!50!white] ( \x, \y ) circle ( 0.1 );
\foreach \x in { 0.225, 0.425, 0.625, 0.825 }
  \foreach \y in { 0.125, 0.325, ..., 2.125 }
    \draw[fill=gray] ( \x, \y ) circle ( 0.1 );

% Swirl flow
\draw ( -2.95, 2 ) -- ++( 0, 2.7 ) -- ++( 0.95, 0 );
\draw ( -3.05, 2 ) -- ++( 0, 2.8 ) -- ++( 1.05, 0 );
\draw ( 2.95, 2 ) -- ++( 0, 2.7 ) -- ++( -0.95, 0 );
\draw ( 3.05, 2 ) -- ++( 0, 2.8 ) -- ++( -1.05, 0 );

\draw [->] ( 0.5, -2 ) -- ++( -0.3, 0.75 );
\draw [->] ( -0.5, -2 ) -- ++( 0.3, 0.75 );
\node at ( 0.5, -2 ) [below] {Core flow};

\draw [->] ( 3, 1 ) -- ++( 0, 0.75 );
\draw [->] ( -3, 1 ) -- ++( 0, 0.75 );
\node at ( 3, 1 ) [below] {Swirl flow};

\end{tikzpicture}

\caption[Detail schematic of Configuration B]{A cross-sectional view of Configuration B is shown. The core flow reactants enter through ports in the base flange. Stainless steel ball bearings partially fill the plenum chamber and render the core flow spatially uniform. The turbulence generator is located within the plenum and is outlined in \textcolor{red}{red}. The swirl flow reactants enter through separate pipes and are injected into the contoured nozzle through four ports.}

\label{fig:lsbB}

\end{figure}

