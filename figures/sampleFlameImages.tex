\begin{figure}

\begin{subfigure}{\linewidth}
  \centering
  \input{figures/sampleAverageImage}
  \caption{Average CH* chemiluminescence image}
  \label{fig:sampleAverageImage}
\end{subfigure}

\begin{subfigure}{\linewidth}
  \centering
  \input{figures/sampleCenterlineIntensity}
  \caption{Centerline CH* chemiluminescence intensity}
  \label{fig:sampleCenterlineIntensity}
\end{subfigure}

\begin{subfigure}{\linewidth}
  \centering
  \input{figures/sampleAbelImage}
  \caption{Abel deconvoluted half-image}
  \label{fig:sampleAbelImage}
\end{subfigure}

\caption[Sample CH* chemiluminescence data]{These images illustrate the processing of a typical CH* chemiluminescence dataset. The top image is the mean of 100 frames and shows the LSB flame at 9 atm. The flame standoff distance is calculated by locating the inflection point in the smoothed intensity profile (middle). An Abel deconvolution (bottom) can be used to highlight the flame brush and measure the angle of the flame.}

\label{fig:sampleFlameImages}

\end{figure}

