\begin{figure}

\hfill
\begin{subfigure}{0.45\linewidth}
  \centering
  \input{figures/gri-sd-xch-plot}
  \caption{CH concentration}
  \label{fig:gri-sd-xch}
\end{subfigure}
\hfill
\begin{subfigure}{0.45\linewidth}
  \centering
  \input{figures/gri-sd-delxf-plot}
  \caption{Flame thickness}
  \label{fig:gri-sd-delxf}
\end{subfigure}
\hfill

\hfill
\begin{subfigure}{0.45\linewidth}
  \centering
  \input{figures/gri-sd-sl-plot}
  \caption{Laminar flame speed}
  \label{fig:gri-sd-sl}
\end{subfigure}
\hfill
\begin{subfigure}{0.45\linewidth}
  \centering
  \input{figures/gri-sd-t-plot}
  \caption{Flame temperature}
  \label{fig:gri-sd-t}
\end{subfigure}
\hfill

\caption[Comparison of GRI Mech 3.0 and San Diego mechanisms]{The plots above compare various predicted parameters of an atmospheric pressure, laminar methane-air flame at three preheat temperatures from GRI Mech 3.0 and San Diego mechanisms. The results from GRI Mech 3.0 are plotted in solid lines, while those from the San Diego mechanism are plotted in dashed lines.}

\label{fig:gri-sd}

\end{figure}
