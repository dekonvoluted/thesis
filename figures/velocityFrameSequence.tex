\begin{figure}

\centering

\hfill
\begin{subfigure}{0.45\linewidth}
  \input{figures/lvframe0}
  \caption{Frame \#0}
  \label{fig:lvframe0}
\end{subfigure}
\hfill
\begin{subfigure}{0.45\linewidth}
  \input{figures/hvframe0}
  \caption{Frame \#0}
  \label{fig:hvframe0}
\end{subfigure}
\hfill

\hfill
\begin{subfigure}{0.45\linewidth}
  \input{figures/lvframe1}
  \caption{Frame \#1}
  \label{fig:lvframe1}
\end{subfigure}
\hfill
\begin{subfigure}{0.45\linewidth}
  \input{figures/hvframe1}
  \caption{Frame \#1}
  \label{fig:hvframe1}
\end{subfigure}
\hfill

\hfill
\begin{subfigure}{0.45\linewidth}
  \input{figures/lvframe2}
  \caption{Frame \#2}
  \label{fig:lvframe2}
\end{subfigure}
\hfill
\begin{subfigure}{0.45\linewidth}
  \input{figures/hvframe2}
  \caption{Frame \#2}
  \label{fig:hvframe2}
\end{subfigure}
\hfill

\caption[Effect of reference velocity on the flame structure - II]{The sequence of instantaneous images on the left are taken at 21 m/s at 315 K. The images on the right are acquired at 40 m/s with a preheat of 443 K. The flame sheet on the right is noticeably more wrinkled.}

\label{fig:velocityFrameSequence}

\end{figure}

