\begin{figure}

\centering

\hfill
\begin{subfigure}{0.45\linewidth}
  \input{figures/referenceVelocityHighVelPLIFImage}
  \caption{\(\phi\) = 0.90}
  \label{fig:lowEquivalenceRatioPLIFImage}
\end{subfigure}
\hfill
\begin{subfigure}{0.45\linewidth}
  \input{figures/highEquivalenceRatioPLIFImage}
  \caption{\(\phi\) = 1.05}
  \label{fig:highEquivalenceRatioPLIFImage}
\end{subfigure}
\hfill

\hfill
\begin{subfigure}{0.45\linewidth}
  \input{figures/referenceVelocityHighVelPLIFHistogram}
  \caption{\(\mu\) = 463; \(\sigma\) = 187}
  \label{fig:lowEquivalenceRatioPLIFHistogram}
\end{subfigure}
\hfill
\begin{subfigure}{0.45\linewidth}
  \input{figures/highEquivalenceRatioPLIFHistogram}
  \caption{\(\mu\) = 353; \(\sigma\) = 127}
  \label{fig:highEquivalenceRatioPLIFHistogram}
\end{subfigure}
\hfill

\caption[Effect of equivalence ratio on the flame structure]{The images above are instantaneous CH PLIF images of a relatively lean and rich flame at preheated, atmospheric pressure conditions. The histograms show the statistics of the number of detected intensity edges over 300 such frames. These experiments are conducted on Configuration B.}

\label{fig:equivalenceRatioPLIFResults}

\end{figure}

