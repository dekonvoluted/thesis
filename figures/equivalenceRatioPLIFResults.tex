\begin{figure}

\centering

\hfill
\begin{subfigure}{0.45\linewidth}
  \centering
  \input{figures/referenceVelocityHighVelPLIFImage}
  \caption{Instantaneous PLIF image (Low Swirl case)}
  \label{fig:referenceVelocityLowVelPLIFImage}
\end{subfigure}
\hfill
\begin{subfigure}{0.45\linewidth}
  \centering
  \input{figures/highEquivalenceRatioPLIFImage}
  \caption{Instantaneous PLIF image (High Swirl case)}
  \label{fig:highEquivalenceRatioPLIFImage}
\end{subfigure}
\hfill

\hfill
\begin{subfigure}{0.45\linewidth}
  \centering
  \input{figures/referenceVelocityHighVelPLIFHistogram}
  \caption{\(\mu\) = 463; \(\sigma\) = 187}
  \label{fig:referenceVelocityLowVelPLIFHistogram}
\end{subfigure}
\hfill
\begin{subfigure}{0.45\linewidth}
  \centering
  \input{figures/highEquivalenceRatioPLIFHistogram}
  \caption{\(\mu\) = 353; \(\sigma\) = 127}
  \label{fig:highEquivalenceRatioPLIFHistogram}
\end{subfigure}
\hfill

\caption[Effect of Equivalence Ratio on Flame Structure]{FIXME}

\label{fig:equivalenceRatioPLIFResults}

\end{figure}

