\begin{figure}

\centering

\begin{subfigure}{0.45\linewidth}
  \centering
  \input{figures/referenceVelocityHighVelPLIFImage}
  \caption{Instantaneous PLIF image (Low Swirl case)}
  \label{fig:referenceVelocityLowVelPLIFImage}
\end{subfigure}
\hfill
\begin{subfigure}{0.45\linewidth}
  \centering
  \input{figures/highSwirlPLIFImage}
  \caption{Instantaneous PLIF image (High Swirl case)}
  \label{fig:referenceVelocityHighVelPLIFImage}
\end{subfigure}

\begin{subfigure}{0.45\linewidth}
  \centering
  \input{figures/referenceVelocityHighVelPLIFHistogram}
  \caption{\(\mu\) = 463; \(\sigma\) = 187}
  \label{fig:referenceVelocityLowVelPLIFHistogram}
\end{subfigure}
\hfill
\begin{subfigure}{0.45\linewidth}
  \centering
  \input{figures/highSwirlPLIFHistogram}
  \caption{\(\mu\) = 285; \(\sigma\) = 164}
  \label{fig:referenceVelocityHighVelPLIFHistogram}
\end{subfigure}

\caption[Effect of Swirl on Flame Structure]{FIXME}

\label{fig:swirlPLIFResults}

\end{figure}

