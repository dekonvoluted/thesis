\begin{figure}

\centering

\hfill
\begin{subfigure}{0.45\linewidth}
  \input{figures/referenceVelocityHighVelPLIFImage}
  \caption{\(S\) = 0.57}
  \label{fig:lowSwirlPLIFImage}
\end{subfigure}
\hfill
\begin{subfigure}{0.45\linewidth}
  \input{figures/highSwirlPLIFImage}
  \caption{\(S\) = 0.62}
  \label{fig:highSwirlPLIFImage}
\end{subfigure}
\hfill

\hfill
\begin{subfigure}{0.45\linewidth}
  \input{figures/referenceVelocityHighVelPLIFHistogram}
  \caption{\(\mu\) = 463; \(\sigma\) = 187}
  \label{fig:lowSwirlPLIFHistogram}
\end{subfigure}
\hfill
\begin{subfigure}{0.45\linewidth}
  \input{figures/highSwirlPLIFHistogram}
  \caption{\(\mu\) = 285; \(\sigma\) = 164}
  \label{fig:highSwirlPLIFHistogram}
\end{subfigure}
\hfill

\caption[Effect of swirl on the flame structure]{The images above show a low and high swirl flame imaged by CH PLIF. These experiments are performed on Configuration B under preheated, atmospheric pressure conditions.}

\label{fig:swirlPLIFResults}

\end{figure}

