\begin{figure}

\centering

\begin{subfigure}{\linewidth}
  \input{figures/referenceVelocityDistancePlot}
  \caption{Flame standoff distance as a function of the reference velocity}
  \label{fig:referenceVelocityDistance}
\end{subfigure}

\begin{subfigure}{\linewidth}
  \input{figures/referenceVelocityAnglePlot}
  \caption{Flame cone angle as a function of the reference velocity}
  \label{fig:referenceVelocityAngle}
\end{subfigure}

\caption[Effect of reference velocity on the flame location and shape]{The plots above show the effect of changing the reference velocity on the flame location and flame shape. The \textcolor{blue}{blue} curves are from tests conducted using the \(S_{37^\circ}\) swirler at 6 atm, while the \textcolor{red}{red} curves are from tests using the \(S_{45^\circ}\) swirler at 3 atm.}

\label{fig:referenceVelocityResults}

\end{figure}

