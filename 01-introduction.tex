\chapter{Introduction}
\label{ch:introduction}

% This chapter should provide the motivation for the work done on this thesis.

% 1. Motivation
% Motivate the problem at hand - low emissions, balanced with stability
% LSB is known to work well at atmospheric pressure.
% Applications of the LSB are at high pressure.
% Important to know how the flame behaves at those conditions.
% Chemiluminescence is effective somewhat
% A planar imaging technique is needed, like CH PLIF.

% 2. Literature Survey - LSB
% History of the low swirl burner
% Original intent, evolution of the LSB design.
% Key discoveries, etc.

% 3. Literature Survey - CH PLIF
% Advantages over other planar techniques
% Early work done by researchers
% Recent advances in LIF

% 4. Summary of goals/Objectives of this thesis

\section{Motivation}

The need to reduce pollutant emissions, particularly the oxides of nitrogen, \ce{NO_x}, in order to meet increasingly stringent government regulations spurs efforts in the gas turbine industry to seek cleaner, more environment-friendly combustion concepts.
The production rate of \ce{NO_x} is highly temperature dependent and fuel-lean, premixed combustion is a widely employed technique to keep the adiabatic flame temperature under 1800 K.
However, operating a combustor at such lean conditions results in eaker combustion processes that are highly susceptible to perturbations and combustor instabilities.
This highlights the requirements for robust flame stabilization techniques to sustain combustion at ultra-lean conditions.
The Low Swirl Burner (LSB) design addresses these requirements by providing a low \ce{NO_x} combustor that can operate stably at lean equivalence ratios.


