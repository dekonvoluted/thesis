\chapter{Introduction}
\label{ch:introduction}

% This chapter should provide the motivation for the work done on this thesis.

% 1. Motivation
% Motivate the problem at hand - low emissions, balanced with stability
% LSB is known to work well at atmospheric pressure.
% Applications of the LSB are at high pressure.
% Important to know how the flame behaves at those conditions.
% Chemiluminescence is effective somewhat
% A planar imaging technique is needed, like CH PLIF.

% 2. Literature Survey - LSB
% History of the low swirl burner
% Original intent, evolution of the LSB design.
% Key discoveries, etc.

% 3. Literature Survey - CH PLIF
% Advantages over other planar techniques
% Early work done by researchers
% Recent advances in LIF

% 4. Summary of goals/Objectives of this thesis

\section{Motivation}

The need to reduce pollutant emissions, particularly the oxides of nitrogen, \ce{NO_x}, in order to meet increasingly stringent government regulations spurs efforts in the gas turbine industry to seek cleaner, more environment-friendly combustion concepts.
The production rate of \ce{NO_x} is highly temperature dependent and fuel-lean, premixed combustion is a widely employed technique to keep the adiabatic flame temperature under 1800 K.
However, operating a combustor at such lean conditions results in eaker combustion processes that are highly susceptible to perturbations and combustor instabilities.
This highlights the requirements for robust flame stabilization techniques to sustain combustion at ultra-lean conditions.
The Low Swirl Burner (LSB) design addresses these requirements by providing a low \ce{NO_x} combustor that can operate stably at lean equivalence ratios.

% Transplanted text introducing/literature surveying CH PLIF
In typical hydrocarbon-air flames, CH is produced and destroyed rapidly by fast two-body reactions, confining it to the thin heat release zone of the flame.
This has led to the widespread use of using CH as a marker species for the flame front.\cite{2005-vagelopoulos}
CH is formed during the breakup of hydrocarbon fuel molecules\cite{2010-kohler} and is also known to play an important role in the production of prompt \ce{NO_x}.\cite{1971-fenimore}
Hence, it is a minor species of considerable importance to combustion research.

This study primarily utilizes CH PLIF as a flame visualization technique.
Broadly speaking, flame visualiation techniques could be grouped into two categories: techniques that convey spatially integrated information about the flame and techniques that resolve the information spatially.
Schlieren or chemiluminescence imaging of flames fall in the former category, providing line-of-sight averaged information about the flame.
Planar LIF or typical implementations of Rayleigh/Raman scattering are examples of the second category.

PLIF studies of hydrocarbon flames have hitherto focused on the OH radical.
OH is produced in the flame zone and destroyed by relatively slow three-body reactions, causing it to persist and be transported away from the flame and into the product zone.\cite{1990-barlow}
As a result, it does not serve as a direct marker of the flame front.
Instead, the location of the flame is inferred from the sharp gradient in the OH signal as the reactants are converted into products.

Due to the persistence of OH in the products, OH PLIF is somewhat less suited to studying flows with high product recirculation.
In such flows, the presence of OH both in the reactants and in the products weakens the gradient at the flame.
Further, OH radicals can be transported transverse to the flame, making it difficult to detect local flame extinctions.
Nevertheless, researchers have been able to use OH PLIF to study such flames,\cite{1999-kaminski,2005-hult} particularly when the images are enhanced by nonlinear filtering techniques.\cite{2000-malm,2001-abu-gharbieh}

Recent studies\cite{1998-najm} have indicated that the species HCO is a superior indicator of heat release in hydrocarbon flames when compared to CH or OH.
The HCO LIF signal has been demonstrated to correlate well with the heat release rate, with little dependence on equivalence ratio or strain rate.
The last factor, in particular, has been shown to quench the CH PLIF signal\cite{2008-kiefer} in highly strained flames, even when the flame itself is not extinguished.
Unfortunately, the signal levels from HCO LIF are very poor\cite{1998-najm,1998-paul} and are unsuitable for single-shot investigation of hydrocarbon flames.
To overcome this, one study\cite{1998-paul} proposed a simultaneous LIF investigation of \ce{CH2O} and OH with the reasoning that the formation rate of HCO is governed directly by the product of the concentration of these two intermediates.
This method has been used in a number of investigations, despite being experimentally cumbersome.
A later implementation\cite{2009-kiefer} recently demonstrated single-shot HCO PLIF with moderate signal-to-noise ratios by utilizing a novel excitation scheme.


item Nearly resonant PLIF. Excite A(0,0) band, observe A(0,0).
item Excite B(0,0), observe A(0,0)
  subitem Excite Q lines.
  subitem Excite R lines / bandhead.


