\chapter{Introduction}
\label{ch:introduction}

The need to reduce pollutant emissions, particularly the oxides of nitrogen, \ce{NO_x}, is driven by increasing ecological awareness and stringent government regulations.
This spurs efforts in the gas turbine industry to seek cleaner, more environment-friendly combustion concepts.
Several mechanisms have been identified to explain the production of \ce{NO_x} in hydrocarbon-air combustion systems.
Of these, the thermal \ce{NO_x} mechanism discovered by Zel'dovich, is a prominent source of \ce{NO_x} production at the high temperature conditions encountered in typical combustors.
The amount of thermal \ce{NO_x} produced scales exponentially with the adiabatic flame temperature.

Efforts to reduce the flame temperature have led low \ce{NO_x} gas turbine manufacturers to adopt one of two options---Lean Premixed (LP) operation, or Rich-Quench-Lean (RQL) operation.
Of these, ground-based gas turbines used in power generation have tended to favor LP operation as it is conceptually simpler and avoids issues resulting from inhomogeneous mixing of fuel and air.
Further, the ultra-lean operating conditions reduce flame temperature and minimize \ce{NO_x} production.

In practice, 1800 K is considered a limiting value for the flame temperature, ensuring that the thermal \ce{NO_x} production is constrained to a minimum.\cite{1996-glassman}
Operating a combustor at such lean conditions results in weaker combustion processes that are highly susceptible to perturbations and results in combustor instabilities or even flame blow off.
This highlights the requirement for robust flame stabilization techniques that can sustain combustion at ultra-lean conditions.
In their most basic form, flame stabilization techniques work by making the local reactant velocity and the local flame speed equal.
In the context of lean flames, the risk is of the slowly propagating flames to be blown off by the high velocity reactant stream.
Consequently, flame stabilization in gas turbine combustion is brought about either by reducing the local reactant velocity (e.g. by using bluff body flame holders), by boosting the local flame velocity (e.g. by enhancing product recirculation), or by providing continual ignition to the flame (e.g. by using pilot flames).

Swirl-stabilized combustion is a widely used flame stabilization technique in gas turbine applications.\cite{1974-syred,1977-lilley}
It primarily functions by inducing recirculation zones in the flow field that transport heat and radicals from the products into the reactants.
This enhances the flame propagation velocity by increasing reaction rates within the flame, resulting in robust flame stabilization.
However, the recirculation zones are associated with high peak residence times for hot combustion products and are sites of themal \ce{NO_x} production in the combustor.
Nevertheless, swirl-stabilized combustors are ubiquitously employed today in land-based gas turbines used for power generation.

More recent research\cite{1995-bedat} on the Low Swirl Burner (LSB) has identified a potential solution for this problem.
The LSB anchors a lifted flame, reducing the need for high swirl in the flow field.
The lifted, V-shaped flame is stabilized by aerodynamic means which allows for robust operation even at low equivalence ratios.
This weakens the recirculation zones and eliminates pockets of high residence times, resulting in the potential for significantly reduced \ce{NO_x} emissions compared to a similar high-swirl design.

\section{Motivation}

By comparison to atmospheric pressure experiments, high pressure experimental testing of combustion systems is fraught with difficulties.
This is reflected in the comparatively smaller subset of publications that report experimental results from high pressure tests.
The primary source of these difficulties stems from the need for complicated testing facilities to reach and maintain high pressures.
The inherently limited access afforded by pressure vessels makes intrusive methods of data gathering nearly impossible.
As a result, any need for spatially resolved data other than temperature and pressure measurements has to be met by optical diagnostics.

In the context of LSB research, these difficulties have confined much of the published experimental results to ambient conditions.
The eventual application of this technology in gas turbine engines requires high quality data acquired at high pressure conditions.
Ideally, such data will map the velocity field and heat release in the LSB and study their variation with flow conditions.
Since the LSB relies on the velocity field to stabilize its flame, its flame characteristics hold information pertinent to both the velocity field and the heat release distribution within the combustor.
This allows a passive diagnostic such as recording the flame chemiluminescence to be used even at high pressure conditions to observe and record usable data about the LSB flame characteristics.
Such data, acquired at conditions closer to real world gas turbine combustor operating conditions is of particular interest to the gas turbine industry as it can be used for designing better, more robust combustors with low \ce{NO_x} emissions.

The primary flame characteristic of interest is the flame standoff distance, defined as the distance from the flame stabilization point to the inlet of the LSB.
This metric is useful is gauging the stability of the flame and the need for control systems to closely monitor its tendency to flashblack or blow-off.
The standoff distance also relates to the heat load experienced by the injector and consequently affects how often the mechanical components of the LSB will require to be replaced in operation.
Finally, a systemic variation in the location of the flame over a range of flow parameters may indicate potential problems operating the combustor at previously untested conditions.

Quantifying the shape of the flame can complement the information gleaned from the flame standoff measurements.
In case of the V-shaped LSB flame, this can be conveniently obtained by measuring the angle of the flame cone.
Changes in the flame angle affect the length of the flame, which is a design consideration for sizing LSB combustors in gas turbines.

The profile of the flame chemiluminescence along the length of the combustion zone is representative of the local heat release at those locations.
A uniform heat release profile is preferred so as to avoid thermally stressing the combustor at the hot spots.
Further, since \ce{NO_x} production rates are so strongly dependent on temperature, the heat release profile can help forecast emissions performance issues of the combustor, particularly when augmented by knowledge of the local flow velocity (and hence, residence time).
Finally, the heat release map could be incorporated into n-\(\tau\) models to predict the onset of thermo-acoustic instabilities in the combustor.

The primary goal of this research work is to study the flame characteristics of the LSB, such as its location and shape, as a means to learn more about the combustor operation at high pressure conditions.

In case of lean hydrocarbon flames, the primary sources of flame chemiluminescence are OH* (\(A^2\Sigma^+\rightarrow X^2\Pi\) bands, 310 nm), CH* (\(A^2\Delta\rightarrow X^2\Pi\) bands, 430 nm, \(B^2\Sigma^-\rightarrow X^2\Pi\) bands, 390 nm), \ce{C2}* (\(d^3\Pi\rightarrow a^3\Pi\) Swan bands, 470 nm, 550 nm) and the \ce{CO2}* (band continuum, 320--500 nm).
Of these, CH* chemiluminescence has several advantages that make it suitable for this particular study.
First, collection of CH* chemiluminescence is less affected by blackbody radiation from the walls of the combustor, compared to longer wavelength emissions from a species like \ce{C2}*.
Its narrow bandwidth allows one to use a bandpass filter to collect signals from only the wavelengths of interest, further minimizing interference from other light sources.
Using such a narrow bandpass filter for a broad band emitter like \ce{CO2}* would result in rejecting most of the available signal.
CH* chemiluminescence occurs in the visible wavelengths and does not require expensive UV lenses or imaging systems with high quantum efficiencies in UV to record it---as would be needed to image OH* chemiluminescence, for instance.
In typical LSB operation, where the flame is not expected to operate near extinction, CH* chemiluminescence can serve as a reliable indicator of heat release in the combustor.
For all these reasons, CH* chemiluminescence is a suitable technique to image the LSB flame.

Ultimately, the amount of information that can be gathered by imaging the flame chemiluminescence is limited by its spatial resolution.
Since chemiluminescence imaging is integrated over the line of sight, studying the flame brush or the flame structure is beyond its capabilities.
A planar imaging technique such as Planar Laser-Induced Fluorescence (PLIF) is better suited for such applications.

In hydrocarbon flames, species accessible to PLIF are generally minor species in the flame.
PLIF studies of hydrocarbon flames have hitherto focused on the hydroxyl, OH, radical.
However, OH is produced in the flame zone and destroyed by relatively slow three-body reactions, causing it to persist and be transported away from the flame and into the product zone.\cite{1990-barlow}
As a result, it does not serve as a direct marker of the flame front.
Instead, the location of the flame is inferred from the sharp gradient in the OH signal as the reactants are converted into products.

The persistence of OH in the products makes OH PLIF somewhat less suited to studying flames in flows with high product recirculation.
In such flows, the presence of OH in both the reactants and the products weakens the gradient at the flame.
Further, since OH radicals could be transported transverse to the flame, its presence or absence serves as an unreliable indicator of local flame extinctions.
Nevertheless, researchers have been able to use OH PLIF successfully\cite{1999-kaminski,2005-hult} to study such flames, particularly when the images are enhanced by nonlinear filtering techniques.\cite{2000-malm,2001-abu-gharbieh}

This study utilizes CH PLIF as the flame visualization technique.
CH is produced and destroyed rapidly by fast two-body reactions, confining it to the thin heat release zone of the flame.
This makes it suitable for use as a marker species for the flame front.\cite{2005-vagelopoulos}
CH is formed during the breakup of hydrocarbon fuel molecules\cite{2010-kohler} and is also known to play an important role in the production of prompt \ce{NO_x}.\cite{1971-fenimore}
Hence, it is a minor species of considerable importance to combustion research.
This leads us to the second motivation for this study---to examine the use of CH PLIF as a flame imaging technique in combustion systems and further, to use it to image and study the LSB flame.

The use of CH PLIF to study lean hydrocarbon flames has been difficult in the past due to several issues.
First, the concentration of the CH species in hydrocarbon flames rapidly declines with equivalence ratio, making high quality imaging of the flame front at lean conditions challenging.
Further, the implementation techniques in the past have suffered from a host of problems ranging from elastic scattering interference to saturation issues leading to diminished signal-to-noise ratios.
However, a recent implementation by Li et al.\cite{2007-li-a} has managed to overcome these issues and has been demonstrated to image moderately lean flames with good fidelity.

Recent studies\cite{1998-najm} have indicated that the formyl species HCO is a superior indicator of heat release in hydrocarbon flames when compared to CH or OH.
The HCO LIF signal has been demonstrated to correlate well with the heat release rate, with little dependence on equivalence ratio or strain rate.
The last factor, in particular, has been shown to quench the CH PLIF signal\cite{2008-kiefer} in highly strained flames, even when the flame itself is not extinguished.
Unfortunately, the signal levels from HCO LIF are very poor\cite{1998-najm,1998-paul} and are unsuitable for single-shot investigation of hydrocarbon flames.
To overcome this, one study\cite{1998-paul} proposed a simultaneous LIF investigation of formaldehyde, \ce{CH2O}, and OH with the reasoning that the formation rate of HCO is governed directly by the product of the concentration of these two intermediates.
This method has been used in a number of investigations,\cite{2006-ayoola} despite being experimentally cumbersome.
A more recent implementation\cite{2009-kiefer}, published after the initiation of the present effort, has demonstrated single-shot HCO PLIF with moderate signal-to-noise ratios by utilizing a novel excitation scheme.
Follow up studies applying this technique in other hydrocarbon flames are awaited.

\section{Literature Review}

\subsection{Low Swirl Burner}
\label{subsec:literature-review-lsb}

The LSB is a relatively new combustion technology and as such has a brief history.
Initial interest in low swirl combustion was primarily motivated by its ability to stabilize a freely propagating turbulent flame.\cite{1992-chan}
As a result, initial designs of the LSB (which at the time used tangential jets to produce swirl) were pursued by B{\'edat} and Cheng\cite{1995-bedat,1995-cheng} as test beds for studying 1-D, planar turbulent flames.
Several subsequent studies\cite{2000-plessing,2001-shepherd,2002-cheng,2002-shepherd,2004-kortschik,2005-degoey,2007-bell} utilized this behavior and investigated fundamental turbulent flame structure and propagation in the jet LSB.
Simultaneously, the discovery of its ability to achieve low \ce{NO_x} emissions prompted interest in commercial applications of the LSB, such as in industrial furnaces and boilers.\cite{1998-yegian,2000-cheng,2002-littlejohn}

The current form of the LSB (as used in this thesis) using vanes to generate swirl was originally modified from a typical production swirl injector used in gas turbine combustors.
The results of testing this new design were published by Johnson et al.\cite{2005-johnson}
The design elements of the new injector---now called the Low Swirl Injector (LSI)---were tuned in an atmospheric pressure test rig using LBO and flame location as the criteria.
The atmospheric tests were conducted with preheated reactants at up to 650 K.
The more interesting results from the work came from high pressure, high preheat tests (15 atm, 700 K) in a test rig with limited optical access.
The researchers measured a dramatic (50\%) reduction in the \ce{NO_x} emissions by switching from the original (``High'' Swirl Injector) to the new low swirl design.
The emissions performance was also noted by Nazeer et al.\cite{2006-nazeer}

Subsequent studies by Cheng et al.\cite{2006-cheng,2008-cheng-a} explored the characteristic velocity field in the LSB using PIV and discovered self-similar behavior that implied that the flame location was unaffected by the mass flow rate of the reactants.
This led to further insights into the flame stabilization mechanism used by the LSB.
These results will be revisited in Chapter \ref{ch:background} in greater detail.

The effects of using an enclosure to contain the combustion zone were explored by Cheng et al.\cite{2008-cheng-c} who found scaling criteria for minimizing the effect of the enclosure on the flame stabilization location. 
More recent work has tended to focus on the use of various fuels such as hydrogen mixtures\cite{2008-cheng-b} with and without dilution\cite{2007-littlejohn}, landfill gas\cite{2008-cheng-a,2009-cheng} and syngas\cite{2010-littlejohn}.

Relatively little research has focused on the flame location and other characteristics and studied their variation at gas turbine relevant conditions.
Plessing et al.\cite{2000-plessing} and Petersson et al.\cite{2007-petersson} have presented planar images of the LSB flame, but have been confined working with non-preheated, atmospheric flames at low flow rates.
Of these, Plessing et al. used a jet-LSB design and imaged the resulting flame with OH Laser-Induced Predissociative Fluorescence to calculate turbulent burning velocities.
Petersson et al. studied a vane-LSB design that is slightly modified from the one tested by Cheng and co-workers and used a bevy of techniques, including OH PLIF to study the turbulent flame.
The OH PLIF images were used to extract mean reaction progress variable contours for comparison to and validation of LES models.
Although their test conditions and burner geometry were different, their results were consistent with the ones published by Cheng et al.
These are notable for being some of the few works that afford us a look at the flame structure in the LSB with good spatial resolution.

\subsection{CH PLIF Implementations}

Historically, CH was the first species to be detected using LIF in a flame.\cite{1973-barnes}
Early attempts\cite{1981-verdieck,1986-allen} to excite the CH layer used variations of short-pulsed, YAG-pumped dye laser output targeting transitions in one branch of the \(A^2\Delta\leftarrow X^2\Pi\) (0,0) band and observing resulting fluorescence in the same band, but at a different rotational branch.
These methods relied on the strong absorption of the \(A-X\) bands to generate high signal values, but suffered from interference from elastic scattering.
Further, the short pulsewidth and narrow spectral bandwidths of the excitation sources quickly saturated the transition being pumped, limiting the amount of LIF signal measured.

Namazian et al.\cite{1986-namazian} and Schefer et al.\cite{1994-schefer} had better success at overcoming interference issues by exciting the \(A-X\) (0,0) band, but observing fluorescence from the (0,1) band.
Another similarly non-resonant technique was proposed by Paul et al.\cite{1994-paul} who excited the \(A^2\Delta\leftarrow X^2\Pi\) (1,0) band and observed resulting fluorescence from the (1,1) and (0,0) bands.
These approaches provide good separation between the excitation and emission wavelengths, but are hampered by the spectroscopic properties of the CH system---which will be explored further in Section \ref{subsec:background-ch-plif-process}---which disfavor radiative transitions in the non-diagonal (0,1) or (1,0) bands.
Further, Namazian et al.'s scheme suffers from interference due to Raman scattering of the excitation beam by the fuel species, which overlaps the (0,1) band fluorescence.

Carter and several others\cite{1998-carter,1999-watson,2000-watson,2000-donbar,2000-han,2002-kothnur,2003-han-a,2003-han-b,2003-sutton} pumped the \(B^2\Sigma^-\leftarrow X^2\Pi\) (0,0) band and utilized fast electronic transfer from the \(B^2\Sigma^-\), \(v=0\) to populate the \(A^2\Delta\), \(v=0,1\) levels.
This way, they could observe the strong emission from \(A^2\Delta\rightarrow X^2\Pi\) (1,1) and (0,0) bands.
This method overcame the interference issues by providing sufficient spacing between the excitation and emission wavelengths, but suffered from saturation issues due to the short pulsewidth of the excitation sources.
Further, at high laser irradiance, the group recorded noticeable interference from fuel LIF.

Li et al.\cite{2007-li-a,2007-li-b,2007-kiefer} investigated the use of an alexandrite laser\cite{2004-li} to improve upon the previous excitation scheme by targeting the R-bandhead of the \(B^2\Sigma^-\leftarrow X^2\Pi\) (0,0) transition with an excitation beam having a much longer pulse duration than Nd:YAG pumped dye lasers.
This excitation scheme offers several advantages over previous implementations.
First, it inherits the large spacing between the excitation and emission wavelengths and reduced interference issues from Carter et al.'s implementation.
Next, by using a long pulsed laser beam, it overcomes saturation issues.
In fact, the researchers note that the pulsewidth is long enough to allow the same CH molecule to go through the excitation-deexcitation sequence several times, boosting signal output.
This aspect of the implementation is further enhanced if the laser is operated in multimode, with a large spectral bandwidth, allowing the laser to target several lines near the R-bandhead.
The resultant improvement in signal-to-noise makes this technique suitable to study even low equivalence ratio hydrocarbon flames.
This is the excitation scheme that is used in this study.

\section{Objectives and Overview}

To summarize, this thesis aims to investigate the behavior of the LSB flame at gas turbine-like conditions by studying its characteristics and quantifying their dependence on various flow and geometric parameters.
Flame characteristics of interest to this study include the flame location, shape and structure.
The investigated parameters are combustor pressure, preheat temperature, reference velocity, equivalence ratio and the swirler vane angle.
Studying the effect of these parameters on the LSB flame will allow us to validate atmospheric pressure/low preheat models of LSB flame stabilization at conditions more relevant to gas turbine operation.
Further, investigating the sensitivity of the flame characteristics to flow and geometric parameters will extend our understanding of the physical processes responsible for LSB flame stabilization.
The results will also aid in designing more robust LSB configurations for use in future gas turbine engines.

Parallel to this, the current work will detail the development of a CH PLIF imaging system to study the structure of lean hydrocarbon flames in preheated combustors.
As a planar imaging technique, this will improve significantly on the spatial and temporal resolution capabilities of other flame imaging techniques like chemiluminescence imaging.
The imaging system will be demonstrated on a laminar flame setup, as well as on the LSB.
The intensity of the CH LIF signal will be modeled to predict its variation with pressure, temperature, and reactant composition.
While the thesis retains its focus on methane-air combustion, mixtures of alkanes and syngases will also be examined by the model for the feasibility of studying the flame with CH PLIF.

The first half of Chapter \ref{ch:background}  provides a brief background discussing previously reported results from LSB investigations conducted by other researchers.
The section discusses the models developed to explain the behavior of the LSB flame and the flow field.
The second half of the chapter discusses two approaches to modeling the CH PLIF signal intensity.
First, a simple two-level model is described, followed by a physical description of the CH PLIF process which highlights the need for a more complex model to calculate the signal intensity.
The chapter concludes with the description of a four-level model of that will predict the CH PLIF intensity with higher fidelity.
Details of the experimental facility and aparatus used to study the LSB and develop the CH PLIF imaging system are presented in Chapter \ref{ch:experimental}.
Chapter \ref{ch:chplif} presents the results and validation of the CH PLIF signal models developed in this theis.
Chapter \ref{ch:lsb} presents results and discussion of the flame characteristics of the LSB acquired at high pressure conditions, along with flame structure images acquired at atmospheric conditions using CH PLIF.
Finally, conclusions drawn from the discussions in Chapters 4--5 and suggestions for future work are presented in Chapter \ref{ch:conclusions}.

